\subsection{Hàm sinh cơ bản}

\begin{tcolorbox}[breakable]
    \begin{baitoan}\label{pb:w02:04}
        Xét $\{a_n\}_{n=0}^\infty$ là dãy số được xác định bởi $$a_0 = 1 \text{ và }\sum\limits_{i=0}^n a_ia_{n-i} = 1,\,\forall n \in \mathbb{Z^+}.$$
        \begin{enumerate}
            \item[(a)] {Gọi $\displaystyle F(x) = \sum\limits_{i=0}^\infty a_ix^i$ là hàm sinh của dãy $\{a_n\}_{n=0}^\infty$. Chứng minh rằng $F(x) = \dfrac{1}{\sqrt{1-x}}$.}
            \item[(b)] {Chứng minh rằng $a_n = \dfrac{(2n-1)!!}{2^n \cdot n!} = \dfrac{1 \cdot 3 \cdot 5 \cdots (2n-1)}{2^n \cdot n!}$.} 
        \end{enumerate}
    \end{baitoan}
\end{tcolorbox}

\textbf{Lời giải. }

\begin{enumerate}
    \item[(a)] {Ta có $\displaystyle F^2(x) = \left(\sum\limits_{i=0}^\infty a_ix^i\right)\left(\sum\limits_{i=0}^\infty a_ix^i\right) = \sum\limits_{n=0}^\infty \left(\sum\limits_{i=0}^n a_ia_{n-i}\right)x^n = \sum\limits_{n=0}^\infty x^n = \lim\limits_{n\to\infty} \dfrac{1-x^n}{1-x} = \dfrac{1}{1-x}$. Suy ra $F(x) = \dfrac{1}{\sqrt{1-x}}$.}
    \item[(b)] {Ta có $F(x) = (1-x)^{-1/2},\,F'(x) = \dfrac{1}{2}(1-x)^{-3/2},\,F''(x) = \dfrac{1 \cdot 3}{2^2}(1-x)^{-5/2},$ từ đó quy nạp được $$F^{(n)}(x) = \dfrac{1 \cdot 3 \cdots (2n-1)}{2^n}(1-x)^{-(2n+1)/2},\,\forall n\in \mathbb{Z^+}.$$
    
    Áp dụng khai triển Maclaurin, ta có $$F(x) = F(0) + \dfrac{F'(0)}{1!}x + \dfrac{F''(0)}{2!}x^2 + \cdots + \dfrac{F^{(n)}(0)}{n!}x^n + \cdots.$$
    
    So sánh hệ số của $x^n$, suy ra $$a_n = \dfrac{F^{(n)}(0)}{n!} = \dfrac{1 \cdot 3 \cdot 5 \cdots (2n-1)}{2^n \cdot n!},\,\forall n\in \mathbb{Z^+}.$$} 
\end{enumerate}

% \begin{tcolorbox}[breakable]
%     \begin{baitoan}\label{pb:w03:01}
%         Có bao nhiêu cách xếp một giỏ gồm $n$ trái cây gồm (táo, chuối, cam,đào), sao cho số táo phải là lẻ, số chuối chia hết cho $k \geq 2$, $n_o \leq \text{số cam} \leq N_o$ và $n_p \leq \text{số đào} \leq N_p$.
%     \end{baitoan}
% \end{tcolorbox}

% \textbf{Lời giải. }

% \begin{tcolorbox}[breakable]
%     \begin{baitoan}\label{pb:w05:06}
%         Giải bài toán \ref{pb:w05:05} bằng phương pháp hàm sinh.
%     \end{baitoan}
% \end{tcolorbox}

% \textbf{Lời giải. }

% \begin{tcolorbox}[breakable]
%     \begin{baitoan}[\cite{shahriari2021invitation}, p. 57]\label{pb:w08:19}
%         Bạn muốn sắp xếp $17$ cuốn sách trên kệ của 1 hiệu sách. Kệ sách dành riêng cho $3$ tiểu thuyết Toni Morrison được xuất bản từ năm 1977 \& 1987: Song of Solomon, Tar baby, \& Beloved. Bạn có nhiều bản sao của mỗi cuốn, nhưng trên kệ bạn muốn có 1 số chẵn Song of Solomon, ít nhất $3$ bản sao Tar Baby, \& nhiều nhất $4$ bản sao Beloved. Có bao nhiêu cách sắp xếp khác nhau có thể thực hiện? (Các bản sao của cùng 1 cuốn sách không nhất thiết phải nằm cạnh nhau.)
%     \end{baitoan}
% \end{tcolorbox}

% \textbf{Lời giải. }