\subsection{Nhận xét chung}
~

Nhận xét tổng quan của em về đề thi, cơ bản cả hai phân môn không có sự thay đổi nhiều về format so với kỳ thi năm 2023. Tuy nhiên, theo quan điểm của em, đề thi 2024 có phần dễ chịu hơn so với đề thi năm trước đó, rút ra được từ phần làm bài của bản thân em trong phòng thi cũng như cut-off điểm cho từng mức giải. 

\textbf{Đại số. } Đánh giá mức độ khó (chủ quan) giảm dần B.4 $-$ B.5 $-$ B.2 $-$ B.3 $-$ B.1.
    \begin{enumerate}
        \item[] {\textbf{Bài B.1.} Hạng và định thức của ma trận, không gian nghiệm của hệ phương trình thuần nhất.}
        \item[] {\textbf{Bài B.2.} Chéo hóa ma trận, phương trình ma trận.}
        \item[] {\textbf{Bài B.3.} Chuyển trạng thái Xích Markov.}
        \item[] {\textbf{Bài B.4.} Tổ hợp đếm.}
        \item[] {\textbf{Bài B.5.} Không gian nghiệm của hệ phương trình thuần nhất, tính chia hết của số nguyên.}
    \end{enumerate}

\textbf{Giải tích. } Đánh giá mức độ khó (chủ quan) giảm dần B.4 $-$ B.5 $-$ B.1 $-$ B.3 $-$ B.2.
    \begin{enumerate}
        \item[] {\textbf{Bài B.1.} Giới hạn dãy số, chuỗi số.}
        \item[] {\textbf{Bài B.2.} Tính liên tục và khả vi của hàm số.}
        \item[] {\textbf{Bài B.3.} Sự tương giao của đồ thị hàm số, cực trị hàm số.}
        \item[] {\textbf{Bài B.4.} Bài toán tổng hợp về hàm số.}
        \item[] {\textbf{Bài B.5.} Ứng dụng của tích phân trong tính diện tích.}
    \end{enumerate}

Trong bài báo cáo này, các bài mà em chưa làm lần lượt là A.5c (Đại số) và B.4ab, A.4a, A.5bc (Giải tích).