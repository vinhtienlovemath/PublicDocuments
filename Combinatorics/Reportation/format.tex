\usepackage[top=1.5cm, bottom=2.5cm, left=1cm, right=1cm]{geometry}
\usepackage[utf8]{vietnam}
\usepackage{array}
\usepackage{caption}
% \captionsetup[table]{name=}
\usepackage{biblatex}
\addbibresource{references.bib}
\usepackage{amsmath,amsthm,amsfonts,amssymb,amscd, fancyhdr, color, comment, graphicx, environ}
\usepackage{mathtools}
\usepackage[
  bookmarksopen,
  bookmarksdepth=2,
  breaklinks=true
]{hyperref}
\usepackage{float}
\usepackage{lastpage}
\usepackage[dvipsnames]{xcolor}
\usepackage{xcolor}
\usepackage[framemethod=TikZ]{mdframed}
\usepackage{enumerate}
\usepackage[shortlabels]{enumitem}
\usepackage{fancyhdr}
\usepackage{indentfirst}
\usepackage{listings}
\usepackage{sectsty}
\usepackage{thmtools}
\usepackage{shadethm}
\usepackage{setspace}
\usepackage{footnote}
\usepackage{nicematrix}
\usepackage{multirow}
\usepackage{booktabs}
\usepackage{amsmath,amsxtra,amssymb,latexsym, amscd,amsthm}
\usepackage{import}
\usepackage{graphicx,todo}
\usepackage{xstring}
\usepackage{mathrsfs}
\usepackage{tikz}
\usetikzlibrary{calc,intersections}
\usetikzlibrary{arrows}
\usepackage{tkz-base}
\usepackage{tkz-euclide}
\usepackage{tkz-tab}
\usepackage{import}
\usepackage{multicol}
% \usepackage{listings}
\usepackage{lscape}
\usepackage{booktabs}
\usepackage{tabto}
\usepackage[most]{tcolorbox}
\usepackage{tabularx}
\newcolumntype{C}[1]{>{\centering\arraybackslash}p{#1}}

% \usepackage{mathrsfs}
% \usepackage{newtxmath}
% \usepackage{mathptmx} % dùng để chỉnh font Latex
% \usepackage[math-style=ISO]{unicode-math} % dùng để chỉnh font Latex
% \setmathfont{TeX Gyre Termes Math} % dùng để chỉnh font Latex

%%%%%%%%%%%%%%%%%%%%
% Code setup
\usepackage{fancyvrb} % C++
\usepackage{environ} % added  <<<<<<<<<<<<
\NewEnviron{TCBx}{ % footnotes ouside the box
    \begin{savenotes}
        \begin{tcolorbox}
                \BODY   
        \end{tcolorbox}
    \end{savenotes}}
\hypersetup{
    colorlinks=true,
    linkcolor=blue,
    filecolor=magenta,      
    urlcolor=blue,
}

% Environment setup
\mdfsetup{skipabove=\topskip,skipbelow=\topskip}
\newrobustcmd\ExampleText{%
  An \textit{inhomogeneous linear} differential equation has the form
  \begin{align}
    L[v] = f,
  \end{align}
  where $L$ is a linear differential operator, $v$ is the dependent
  variable, and $f$ is a given non-zero function of the independent
  variables alone.
}
\mdfdefinestyle{theoremstyle}{%
  linecolor=black,linewidth=1pt,%
  frametitlerule=true,%
  frametitlebackgroundcolor=gray!20,
  innertopmargin=\topskip,
}
\mdtheorem[style=theoremstyle]{Problem}{Problem}
\newenvironment{Solution}{\textbf{Solution.}}

\definecolor{codegreen}{rgb}{0,0.6,0}
\definecolor{codegray}{rgb}{0.5,0.5,0.5}
\definecolor{codepurple}{rgb}{0.58,0,0.82}
\definecolor{backcolour}{rgb}{0.95,0.95,0.92}

\lstdefinestyle{mystyle}{
  backgroundcolor=\color{backcolour},   
  commentstyle=\color{codegreen},
  keywordstyle=\color{magenta},
  numberstyle=\tiny\color{codegray},
  stringstyle=\color{codepurple},
  basicstyle=\ttfamily\footnotesize,
  breakatwhitespace=false,         
  breaklines=true,                 
  captionpos=b,                    
  keepspaces=true,                 
  numbers=left,                    
  numbersep=5pt,                  
  showspaces=false,                
  showstringspaces=false,
  showtabs=false,                  
  tabsize=2
}
\lstset{style=mystyle}
%%%%%%%%%%%%%%%%%%%%

%%%%%%%%%%%%%%%%%%%%
% Fill in the appropriate information below
\newcommand{\norm}[1]{\left\lVert#1\right\rVert}     
\newcommand\course{XXXX0000}        % <-- course name   
\newcommand\hwnumber{0}             % <-- homework number
\newcommand\Information{Someone}    % <-- personal information
%%%%%%%%%%%%%%%%%%%%

%%%%%%%%%%%%%%%%%%%%
% Page setup
%---------------
\pagestyle{fancy}
\headheight 35pt
% \lhead{\today}
% \rhead{\includegraphics[width=2.5cm]{UMT_1.jpg}}
% \lfoot{}
% \pagenumbering{arabic}
% \cfoot{\small\thepage}
% \rfoot{}
\headsep 1.2em
\renewcommand{\baselinestretch}{1.25}
%---------------
\fancyhead{}
\fancyhead[L]{
  \begin{tabular}[c]{@{}c@{}}
    \includegraphics[width=2.75cm]{Logo/UMT_1.jpg}
  \end{tabular}
  \fontsize{10}{8}\selectfont
  \begin{tabular}[c]{@{}l@{}}
    \textbf{Trường Đại học Quản lý và Công nghệ TP. HCM} \\
    \textbf{Khoa Công nghệ}
  \end{tabular}
}
\fancyhead[R]{
  \begin{tabular}{l}
    \fontsize{10}{8}\selectfont
    \textbf{Phan Vĩnh Tiến}
  \end{tabular}
}
\fancyfoot{}
\fancyfoot[R]{
  {\small\nouppercase{\fontsize{10}{8}\selectfont \textbf{\rightmark}}}}
\fancyfoot[L]{\parbox[t]{0.4\linewidth}{
  {\small\nouppercase{\fontsize{10}{8}\selectfont \textbf{\leftmark}}}}}
\fancyfoot[C]{{\small\nouppercase{\fontsize{10}{8}\selectfont \textbf{Trang {\thepage}/\pageref{LastPage}}}}}
%%%%%%%%%%%%%%%%%%%%

%%%%%%%%%%%%%%%%%%%%
% Tcolorbox
\BeforeBeginEnvironment{tcolorbox}{\savenotes}
\AfterEndEnvironment{tcolorbox}{\spewnotes}
%%%%%%%%%%%%%%%%%%%%

%%%%%%%%%%%%%%%%%%%%
% New definition
\renewcommand{\labelenumi}{\alph{enumi})}
\newcommand{\Z}{\mathbb Z}
\newcommand{\R}{\mathbb R}
\newcommand{\Q}{\mathbb Q}
\newcommand{\NN}{\mathbb N}
\newcommand{\PP}{\mathbb P}
\DeclareMathOperator{\Mod}{Mod} 
\renewcommand\lstlistingname{Algorithm}
\renewcommand\lstlistlistingname{Algorithms}
\def\lstlistingautorefname{Alg.}
\newtheorem*{theorem}{Theorem}
\newtheorem*{lemma}{Lemma}
\newtheorem{case}{Case}
\newcommand{\assign}{:=}
\newcommand{\infixiff}{\text{ iff }}
\newcommand{\nobracket}{}
\newcommand{\backassign}{=:}
\newcommand{\tmmathbf}[1]{\ensuremath{\boldsymbol{#1}}}
\newcommand{\tmop}[1]{\ensuremath{\operatorname{#1}}}
\newcommand{\tmtextbf}[1]{\text{{\bfseries{#1}}}}
\newcommand{\tmtextit}[1]{\text{{\itshape{#1}}}}
%%%%%%%%%%%%%%%%%%%%

%%%%%%%%%%%%%%%%%%%%
\newcommand*{\mkbibnextfootnotetext}[1]{\nextfootnotetext{\blxmkbibnote{foot}{#1}}}
\DeclareCiteCommand{\footcitetext}[\mkbibnextfootnotetext]
  {\usebibmacro{prenote}}
  {\usebibmacro{citeindex}%
    \usebibmacro{cite}}
  {\multicitedelim}
  {\usebibmacro{cite:postnote}}
%%%%%%%%%%%%%%%%%%%%

%%%%%%%%%%%%%%%%%%%% 
% Custom matrix
\makeatletter
\renewcommand*\env@matrix[2][1.0]{%
  \edef\arraystretch{#1}%
  \hskip -\arraycolsep
  \let\@ifnextchar\new@ifnextchar
  \array{#2}
}
\makeatother
%%%%%%%%%%%%%%%%%%%%

%%%%%%%%%%%%%%%%%%%%
% Some definitions
\theoremstyle{definition}
\allowdisplaybreaks
\newtheorem{dinhly}{Định lý}
\newtheorem{dinhnghia}{Định nghĩa}
\newtheorem{vidu}{Ví dụ}
\newtheorem{hequa}{Hệ quả}
\newtheorem{bode}{Bổ đề}
\newtheorem{baitoan}{Bài toán}
\newtheorem{luuy}{Lưu ý}
\newtheorem{nhanxet}{Nhận xét}
%%%%%%%%%%%%%%%%%%%%

%%%%%%%%%%%%%%%%%%%%
% Create files inside Latex file
\begin{filecontents}{references.bib}
  @book{shahriari2021invitation,
    title={An invitation to combinatorics},
    author={Shahriari, Shahriar},
    year={2021},
    publisher={Cambridge University Press}
  }
\end{filecontents}
%%%%%%%%%%%%%%%%%%%%

%%%%%%%%%%%%%%%%%%%%
% Stirling numbers
\newcommand{\genstirlingI}[3]{%
	\genfrac{[}{]}{0pt}{#1}{#2}{#3}%
}
\newcommand{\genstirlingII}[3]{%
	\genfrac{\{}{\}}{0pt}{#1}{#2}{#3}% 
}
\newcommand{\stirlingI}[2]{\genstirlingI{}{#1}{#2}}
\newcommand{\dstirlingI}[2]{\genstirlingI{0}{#1}{#2}}
\newcommand{\tstirlingI}[2]{\genstirlingI{1}{#1}{#2}}
\newcommand{\stirlingII}[2]{\genstirlingII{}{#1}{#2}}
\newcommand{\dstirlingII}[2]{\genstirlingII{0}{#1}{#2}}
\newcommand{\tstirlingII}[2]{\genstirlingII{1}{#1}{#2}}
%%%%%%%%%%%%%%%%%%%%

%%%%%%%%%%%%%%%%%%%% % Indent & Line-spacing
\setlength\parindent{0pt}
\setlist{nolistsep}
\def\multiset#1#2{\ensuremath{\left(\kern-.3em\left(\genfrac{}{}{0pt}{}{#1}{#2}\right)\kern-.3em\right)}}
\usepackage{breakurl}
%%%%%%%%%%%%%%%%%%%%

