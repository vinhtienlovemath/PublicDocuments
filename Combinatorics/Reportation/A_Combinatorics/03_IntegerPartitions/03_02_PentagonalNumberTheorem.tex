\subsection{Định lý số ngũ giác}
% \begin{tcolorbox}[breakable]
%     \begin{baitoan}[\cite{shahriari2021invitation}, P.7.2.10, p. 245]\label{pb:w08:20}
%         (a) Xét các hình ngũ giác trong {\sf Hình 7.7: Các số ngũ giác $1,5,12,22$}. Chứng minh rằng, nếu chúng ta tiếp tục chuỗi này, hình $k$th sẽ có đúng $\dfrac{k(3k - 1)}{2}$ chấm. (b) Nhân chuỗi các số ngũ giác với $3$. Các số này có liên quan đến các số tam giác $\dfrac{n(n + 1)}{2}$ không? Như thế nào? (c) Xét các số nguyên có dạng $\dfrac{k(3k\pm1)}{2}$ với 1 số $k\in\mathbb{N}$. Các số này liên quan đến các số ngũ giác tổng quát như thế nào?
%     \end{baitoan}
% \end{tcolorbox}

% \textbf{Lời giải. }

% \begin{tcolorbox}[breakable]
%     \begin{baitoan}[\cite{shahriari2021invitation}, P.7.2.17, p. 246]\label{pb:w08:21}
%         Cho $n$ là 1 số ngũ giác tổng quát khác không. Giả sử $n = \dfrac{n(3k\pm1)}{2}$. (a) Nếu $k$ chẵn, hãy chứng minh rằng $|{\cal E}| = |{\cal O}| + 1$. (b) Nếu $k$ lẻ, hãy chứng minh rằng $|{\cal O}| = |{\cal E}| + 1$.
%     \end{baitoan}
% \end{tcolorbox}

% \textbf{Lời giải. }