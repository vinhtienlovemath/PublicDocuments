\subsection{Đồ thị cơ bản}

% \begin{tcolorbox}[breakable]
%     \begin{baitoan}[\cite{shahriari2021invitation}, Example 2.30, p. 57]\label{pb:w08:05}
%         Consider $C_4$, a cycle of length $4$, \& label its vertices around the cycle has $1,2,3,4$. Now, vertices $1,3$ are not adjacent, but each is adjacent to both $2,4$. As such $C_4$ is isomorphic to the bipartite graph $K_{2,2}$.
%     \end{baitoan}
% \end{tcolorbox}

% \textbf{Lời giải. }

\begin{tcolorbox}[breakable]
    \begin{baitoan}[\cite{shahriari2021invitation}, P2.2.1, p. 58]\label{pb:w08:06}
        Vào đầu những năm 1970, Giải bóng bầu dục quốc gia bao gồm $2$ hội nghị, mỗi hội nghị có $13$ đội. Các quy tắc của giải đấu quy định rằng trong mùa giải kéo dài $14$ tuần, mỗi đội sẽ chơi $11$ trận với các đội trong hội nghị của mình \& 3 trận với các đội trong hội nghị đối diện. Chứng minh rằng điều này là không thể.
    \end{baitoan}
\end{tcolorbox}

\textbf{Lời giải. }Xem 13 đội trong từng hội nghị là 13 đỉnh trong một đồ thị đơn vô hướng. Mỗi trận đấu giữa hai đội trong nội bộ hội nghị là một cạnh nối hai đỉnh tương ứng trong đồ thị. Khi đó tổng bậc sẽ bằng $13 \cdot 11 = 143$ là số lẻ, mâu thuẫn với định lý Euler. Như vậy ta có điều phải chứng minh.

\begin{tcolorbox}[breakable]
    \begin{baitoan}[\cite{shahriari2021invitation}, P2.2.2, p. 58]\label{pb:w08:07}
        Trong 1 đồ thị đơn hữu hạn, liệu tất cả các bậc của các đỉnh có thể khác nhau không? Hãy đưa ra 1 ví dụ trong đó tất cả các bậc đều khác nhau hoặc chứng minh rằng trong 1 đồ thị đơn hữu hạn, chúng ta luôn có thể tìm thấy ít nhất $2$ đỉnh có cùng bậc.
    \end{baitoan}
\end{tcolorbox}

\textbf{Lời giải. }Giả sử tồn tại đồ thị đơn $n$ đỉnh có bậc các đỉnh đôi một khác nhau. Khi đó tập bậc đúng bằng $\{0,\,1,\,2,\,\ldots,\,n-1\}$. Như vậy tồn tại một đỉnh có bậc $n-1$ (tức được nối với tất cả các đỉnh còn lại) và một đỉnh có bậc $0$ (tức không nối với đỉnh nào). Điều này mâu thuẫn nên luôn tồn tại hai đỉnh có cùng bậc.

\begin{tcolorbox}[breakable]
    \begin{baitoan}[\cite{shahriari2021invitation}, P2.2.3, p. 58]\label{pb:w08:08}
        Bạn được cung cấp 1 đồ thị đơn giản với $n\in\mathbb{N}^\star$ đỉnh. Bạn cũng biết rằng đồ thị là đồ thị hai phần. Số cạnh tối đa có thể có trong đồ thị này là bao nhiêu?
    \end{baitoan}
\end{tcolorbox}

\textbf{Lời giải. }Xét hai phần của đồ thị hai phía $G$ này là $U$ và $V$. Để đồ thị có số cạnh tối đa, thì đồ thị $G$ phải là đồ thị đầy đủ. Gọi $k$ là số đỉnh trong $U$ thì $n-k$ là số đỉnh trong $V$, khi đó số cạnh trong $G$ bằng $k(n-k)$. Áp dụng bất đẳng thức AM-GM thì $k(n-k) \leq \left(\dfrac{k + (n-k)}{2}\right)^2 = \dfrac{n^2}{4}$. Tuy nhiên $k(n-k)$ là số nguyên nên $k(n-k) \leq \left\lfloor\dfrac{n^2}{4}\right\rfloor$.

% \begin{tcolorbox}[breakable]
%     \begin{baitoan}[\cite{shahriari2021invitation}, P2.2.4, p. 58]\label{pb:w08:09}
%         Tôi có 1 đồ thị đơn giản với $47$ đỉnh. Số cạnh tối đa là bao nhiêu? Số cạnh tối đa là bao nhiêu nếu, ngoài ra, tôi biết rằng đồ thị không được kết nối (một đồ thị được gọi là kết nối nếu bạn có thể đi từ bất kỳ đỉnh nào đến bất kỳ đỉnh nào khác bằng cách đi qua các cạnh)?
%     \end{baitoan}
% \end{tcolorbox}

% \textbf{Lời giải. }

% \begin{tcolorbox}[breakable]
%     \begin{baitoan}[\cite{shahriari2021invitation}, P2.2.5, p. 58]\label{pb:w08:10}
%         Lặp lại Bài toán \ref{pb:w08:09} nhưng thay $47$ bằng 1 số nguyên khác $> 1$. Bạn có thấy 1 mô hình không? Giả sử số đỉnh của 1 đồ thị đơn giản là $n$. Giả sử $M_1$ là số cạnh tối đa mà đồ thị có thể có, \& giả sử $M_2$ là số cạnh tối đa có thể có nếu đồ thị không được kết nối. $M_1 - M_2$ là gì?
%     \end{baitoan}
% \end{tcolorbox}

% \textbf{Lời giải. }

% \begin{tcolorbox}[breakable]
%     \begin{baitoan}[\cite{shahriari2021invitation}, P2.2.6, p. 58]\label{pb:w08:11}
%         Mỗi $9$ người dùng có xu hướng $3$ yêu cầu kết bạn trên 1 nền tảng truyền thông xã hội. Có khả năng là mọi người dùng đều nhận được yêu cầu kết bạn từ chính $3$ người mà họ đã gửi yêu cầu không? Nếu số lượng người dùng là $8$, hoặc nói chung là $n\in\mathbb{N}^\star$ thì sao?
%     \end{baitoan}
% \end{tcolorbox}

% \textbf{Lời giải. }

% \begin{tcolorbox}[breakable]
%     \begin{baitoan}[\cite{shahriari2021invitation}, P2.2.7, pp. 58--59]\label{pb:w08:12}
%         Cho $G$ là 1 đồ thị có $7$ đỉnh là Mojdeh, Mehrdokht, Māmak, Marjān, Mehrnāz, Mahshid, \& Marzieh. Có 1 cạnh giữa $2$ đỉnh nếu cả hai là ``bạn bè'' trên nền tảng mạng xã hội mà tất cả họ đều tham gia. Danh sách các kết nối này được đưa ra trên bảng cho phần Khởi động 2.12. Vẽ 1 đồ thị $G$ \& sử dụng nó để trả lời các câu hỏi sau. (a) Bạn có thể tìm thấy 1 đường đi có độ dài $6$ trong $G$ không? Nghĩa là, bạn có thể loại bỏ 1 số đỉnh \&{\tt}hoặc 1 số cạnh để đồ thị còn lại là 1 đường đi có độ dài $6$ không? (Nếu bạn loại bỏ 1 đỉnh, bạn phải loại bỏ tất cả các cạnh liên hợp với đỉnh đó, \& hãy nhớ rằng độ dài của 1 đường đi là số {\rm cạnh} trên đường đi.) (b) Bạn có thể tìm thấy 1 chu trình có độ dài $6$ trong $G$ không? Một lần nữa, điều này có nghĩa là bạn muốn 1 chu trình có độ dài $6$ sau khi có thể loại bỏ 1 số đỉnh \&{\tt/}cạnh. (c) Khoảng cách {\rm} giữa $2$ đỉnh là độ dài của đường đi ngắn nhất giữa chúng. Khoảng cách giữa Māmak \& Marzieh là bao nhiêu? (d) $2$ đỉnh nào cách xa nhau nhất? Tức là, khoảng cách giữa $2$ đỉnh nào là lớn nhất có thể? (e) Bậc tối đa của 1 đỉnh trong $G$ là bao nhiêu? Bậc tối thiểu là bao nhiêu?
%     \end{baitoan}
% \end{tcolorbox}

% \textbf{Lời giải. }

% \begin{tcolorbox}[breakable]
%     \begin{baitoan}[\cite{shahriari2021invitation}, P2.2.8, p. 59]\label{pb:w08:13}
%         $1$ nữ hoàng muốn xây $10$ lâu đài được nối với nhau bằng các con mương. Bà muốn chính xác $5$ con mương theo hình dạng các đường thẳng, \& bà muốn $4$ lâu đài trên mỗi con mương. Các cố vấn của bà đề xuất 1 cấu hình hình ngôi sao với các lâu đài tại các giao điểm của đường thẳng (xem {\sf Hình 2.9: $10$ lâu đài tại giao điểm từng cặp của $5$ con mương thẳng.}). Bà thích ý tưởng đặt các lâu đài tại các giao điểm của các con mương, nhưng quyết định thêm 1 điều kiện mới. Bà muốn $1$ (hoặc thậm chí có thể là $2$) lâu đài được bao quanh bởi các con mương, \& do đó không phải chịu sự tấn công trực tiếp từ bên ngoài. Bạn có thể gửi 1 thiết kế không? (Truyện dân gian toán học có rất nhiều phiên bản khác nhau của câu đố này.)
%     \end{baitoan}
% \end{tcolorbox}

% \textbf{Lời giải. }

% \begin{tcolorbox}[breakable]
%     \begin{baitoan}[\cite{shahriari2021invitation}, P2.2.9, p. 59]\label{pb:w08:14}
%         Vào thời điểm nữ hoàng của Bài toán P2.2.8 xem xét tất cả các thiết kế đã nộp, 1 sự suy thoái kinh tế có nghĩa là việc xây dựng $10$ lâu đài là không khôn ngoan. Quay lại bản vẽ rộng. Vì xây dựng mương rẻ hơn nhiều so với xây dựng lâu đài, nữ hoàng đã yêu cầu thiết kế cho $7$ lâu đài \& $7$ mương. $6$ trong số các mương phải là đường thẳng \& 1 mương phải là đường tròn. Sẽ có $3$ lâu đài trên mỗi mương \& mỗi lâu đài sẽ nằm tại giao điểm của $3$ mương. $1$ trong số các lâu đài sẽ được bao quanh bởi mương \& không ở trong nguy cơ bị tấn công trực tiếp ngay lập tức. Nộp 1 thiết kế.
%     \end{baitoan}
% \end{tcolorbox}

% \textbf{Lời giải. }

% \begin{tcolorbox}[breakable]
%     \begin{baitoan}[\cite{shahriari2021invitation}, P2.2.10, p. 59]\label{pb:w08:15}
%         1 hội nghị bơi lội của trường đại học có $7$ đội. Trong suốt mùa giải, mỗi đội tổ chức $1$ ``cuộc gặp gỡ'', \& các cuộc gặp gỡ được lên lịch vào mỗi cuối tuần cách tuần trong suốt mùa giải. Mỗi cuộc gặp gỡ quy tụ 1 số đội, \& các đội đó tham gia vào tất cả các sự kiện. Có thể thiết kế 1 lịch trình sao cho mỗi cặp đội kết thúc trong cùng 1 cuộc gặp gỡ đúng 1 lần không? Tức là, đội $A$ sẽ tổ chức $1$ cuộc gặp gỡ \& sẽ tham gia 1 vài cuộc gặp gỡ khác \& trong quá trình này phải tham gia cùng 1 cuộc gặp gỡ với mỗi đội khác đúng 1 lần. Nộp 1 kế hoạch. Bạn có thể đảm bảo rằng mỗi cuộc gặp gỡ có cùng số lượng đội tham gia không?
%     \end{baitoan}
% \end{tcolorbox}

% \textbf{Lời giải. }

% \begin{tcolorbox}[breakable]
%     \begin{baitoan}[\cite{shahriari2021invitation}, P2.2.11, p. 59]\label{pb:w08:16}
%         Hội nghị của Bài toán P2.2.10 đã mở rộng \& hiện có $13$ đội. Họ nên làm gì bây giờ? Mỗi đội vẫn có thể tổ chức 1 cuộc gặp gỡ 1 lần (do đó $13$ cuộc gặp gỡ), \& mỗi đội thi đấu với mọi đội khác trong đúng $1$ cuộc gặp gỡ?
%     \end{baitoan}
% \end{tcolorbox}

% \textbf{Lời giải. }

% \begin{tcolorbox}[breakable]
%     \begin{baitoan}[\cite{shahriari2021invitation}, P2.2.12, p. 59]\label{pb:w08:17}
%         Xét đồ thị hoàn chỉnh $K_4$. Đồ thị này có $6$ cạnh. Bạn có thể tô $3$ cạnh màu đỏ \& $3$ cạnh còn lại màu xanh lam theo cách mà đồ thị bao gồm các cạnh màu đỏ đồng cấu với đồ thị bao gồm các cạnh màu xanh lam không? Hoặc là chỉ ra cách thực hiện hoặc chứng minh tại sao điều đó là không thể.
%     \end{baitoan}
% \end{tcolorbox}

% \textbf{Lời giải. }

% \begin{tcolorbox}[breakable]
%     \begin{baitoan}[\cite{shahriari2021invitation}, P2.2.14, p. 59]\label{pb:w08:18}
%         Lặp lại Bài toán P2.2.12 với $K_5$. Tức là, bạn có thể phân hoạch các cạnh của $K_5$ thành $2$ tập hợp sao cho $2$ đồ thị (con) kết quả là đẳng cấu không?
%     \end{baitoan}
% \end{tcolorbox}

% \textbf{Lời giải. }
