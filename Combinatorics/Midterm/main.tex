\documentclass[a4paper, 11pt]{article}

\usepackage[english,vietnamese]{babel} 
% \usepackage[margin=1in]{geometry} % full-width
\usepackage{lmodern} % great font
\renewcommand*\familydefault{\sfdefault}	
\usepackage{geometry}
\geometry{left=10mm,
	right=10mm,
	top=10mm,
	bottom=15mm
} 

% AMS Packages
\usepackage{amsmath}
\usepackage{amsthm}
\usepackage{amssymb}
\usepackage{mathtools}

% Unicode
\usepackage[utf8]{inputenc}
\usepackage{hyperref}
\hypersetup{
	unicode,
%	colorlinks,
%	breaklinks,
%	urlcolor=cyan, 
%	linkcolor=blue, 
	pdfauthor={Author One, Author Two, Author Three},
	pdftitle={A simple article template},
	pdfsubject={A simple article template},
	pdfkeywords={article, template, simple},
	pdfproducer={LaTeX},
	pdfcreator={pdflatex}
}

% Tcolorbox
\usepackage{hyperref}
\usepackage{footnote}
\usepackage{fancyvrb}
\usepackage[most]{tcolorbox}
\usepackage{environ} % added  <<<<<<<<<<<<
\NewEnviron{TCBx}{ % footnotes ouside the box
    \begin{savenotes}
        \begin{tcolorbox}
                \BODY   
        \end{tcolorbox}
    \end{savenotes}}
\hypersetup{
    colorlinks=true,
    linkcolor=blue,
    filecolor=magenta,      
    urlcolor=blue,
}
\BeforeBeginEnvironment{tcolorbox}{\savenotes}
\AfterEndEnvironment{tcolorbox}{\spewnotes}

% Table and Figure setup
\usepackage{caption}
\captionsetup[table]{labelfont=bf,labelsep=period,justification=centering}
\captionsetup[figure]{labelfont=bf,labelsep=period,justification=centering}

% Section's name setup
% \usepackage{titlesec}
% \titleformat{\section}{\normalfont\fontsize{14}{16}\bfseries}{\thesection.}{1em}{}
% \titleformat{\subsection}{\normalfont\fontsize{12}{16}\bfseries}{\thesubsection.}{1em}{}
% \titleformat{\subsubsection}{\normalfont\fontsize{11}{16}\bfseries}{\thesubsubsection.}{1em}{}

% Natbib
\usepackage[sort&compress,numbers,square]{natbib}
\bibliographystyle{mplainnat}

% Tabular
\usepackage{tabularx}
\newcolumntype{C}[1]{>{\centering\arraybackslash}p{#1}}

% % Theorem, Lemma, etc
% \theoremstyle{plain}
% \newtheorem{theorem}{Theorem}
% \newtheorem{corollary}[theorem]{Corollary}
% \newtheorem{lemma}[theorem]{Lemma}
% \newtheorem{claim}{Claim}[theorem]
% \newtheorem{axiom}[theorem]{Axiom}
% \newtheorem{conjecture}[theorem]{Conjecture}
% \newtheorem{fact}[theorem]{Fact}
% \newtheorem{hypothesis}[theorem]{Hypothesis}
% \newtheorem{assumption}[theorem]{Assumption}
% \newtheorem{proposition}[theorem]{Proposition}
% \newtheorem{criterion}[theorem]{Criterion}
% \theoremstyle{definition}
% \newtheorem{definition}[theorem]{Definition}
% \newtheorem{example}[theorem]{Example}
% \newtheorem{remark}[theorem]{Remark}
% \newtheorem{problem}[theorem]{Problem}
% \newtheorem{principle}[theorem]{Principle}

% Some definitions
\theoremstyle{definition}
\allowdisplaybreaks
\newtheorem{dinhly}{Định lý}
\newtheorem{dinhnghia}{Định nghĩa}
\newtheorem{vidu}{Ví dụ}
\newtheorem{hequa}{Hệ quả}
\newtheorem{bode}{Bổ đề}
\newtheorem{baitoan}{Bài toán}
\newtheorem{luuy}{Lưu ý}
\newtheorem{nhanxet}{Nhận xét}
\newtheorem{congthuc}{Công thức}

\usepackage{fancyhdr}
% \fancyfoot{}
\fancyfoot[C]{{\small\nouppercase{\fontsize{10}{8}\selectfont \textbf{Trang {\thepage}/\pageref{LastPage}}}}}

\usepackage{graphicx, color}
\graphicspath{{fig/}}

%\usepackage[linesnumbered,ruled,vlined,commentsnumbered]{algorithm2e} % use algorithm2e for typesetting algorithms
\usepackage{algorithm, algpseudocode} % use algorithm and algorithmicx for typesetting algorithms
\usepackage{mathrsfs} % for \mathscr command

\usepackage{lipsum}

% Author info
\title{\Large \textbf{LỜI GIẢI ĐỀ THI GIỮA KỲ MÔN TỔ HỢP VÀ LÝ THUYẾT ĐỒ THỊ} \\ HỌC KỲ SUMMER 2025}
\author{\normalsize\textbf{Phan Vĩnh Tiến} \\ \normalsize Trường Đại học Quản lý và Công nghệ Thành phố Hồ Chí Minh}
\date{}
% \author{Phan Vĩnh Tiến$^1$\thanks{Author One was partially supported by Grant XXX} \and Author Two$^2$ \and Author Three$^1$}
% \date{
% 	$^1$Organization 1 \\ \texttt{\{auth1, auth3\}@org1.edu}\\%
% 	$^2$Organization 2 \\ \texttt{auth3@inst2.edu}\\[2ex]%
% %	\today
% }

%%%%%%%%%%%%%%%%%%%% Chèn code C/C++
\definecolor{black}{cmyk}{0,0,0,1}             
\definecolor{white}{cmyk}{0,0,0,0}
\definecolor{darkgrey}{cmyk}{0,0,0,0.97}  
\definecolor{greyBackground}{cmyk}{0.0000, 0.0000, 0.0000, 0.0510}
\definecolor{greenComments}{cmyk}{
1.0000, 0.0000, 1.0000, 0.4980}
\definecolor{blueKeywords}{cmyk}{0.8222, 0.8222, 0.0000, 0.2941}
%\definecolor{pinkOtherKeywords}{cmyk}{0.1676, 0.4973, 0.0000, 0.2745}
\definecolor{pinkOtherKeywords}{cmyk}{0.0000, 0.2695, 0.7730, 0.4471}
\definecolor{brownFunctions}{cmyk}{0.0000, 0.1368, 0.3419, 0.5412}
\definecolor{orangeLibraries}{cmyk}{0.0000, 0.3871, 0.9677, 0.1490}
\definecolor{redStrings}{cmyk}{0.0000, 0.8712, 0.8712, 0.3608}
\lstset{
    language=C++,
    numbers=none,
    rulesep=10pt,
    %xleftmargin=12pt,
    %framexleftmargin=-2pt,
    %framexrightmargin=-5pt,
    showtabs=false,
    showspaces=false,
    showstringspaces=false,
    breaklines=true,
    breakatwhitespace=true,
    % backgroundcolor=\color{greyBackground},
    % rulecolor=\color{darkgrey},
    commentstyle=\color{greenComments},
    morecomment=[s][\color{greenComments}]{/*+}{*/},
    morecomment=[s][\color{greenComments}]{/*-}{*/},
    basicstyle=\ttfamily\color{black},
    stringstyle=\color{redStrings},
    % frame=trbl,
    frame=none,
    framesep=1pt,
    numbersep=7pt,
    belowcaptionskip=1\baselineskip,
    columns=fullflexible,
    captionpos=b,
    extendedchars=true,
    keepspaces=true, 
    stepnumber=5, 
    tabsize=4, 
    title=\lstname,
    flexiblecolumns=true, % Allow columns to break within listings
    % Keywords like string, int, false, true ...
    keywordstyle=\color{blue},
    morekeywords={partial, var, value, get, set},
    % Keywords like if/else, switch/case ...
    emphstyle=\color{pinkOtherKeywords},
    emph={if, else, return, throw, switch, case, while, do, while, using, break, continue, for},
    % Collection of your functions
    % emphstyle={[2]\color{brownFunctions}},
    % emph={[2]},
    % Collection of your libraries
    % emphstyle={[3]\color{orangeLibraries}},
    % emph={[3]iostream}
}
\newcolumntype{b}{X}
\newcolumntype{s}{>{\hsize=.75\hsize}X}
%%%%%%%%%%%%%%%%%%%%

% Others
\usepackage{multicol}
\usepackage{enumitem}
\usepackage[ddmmyyyy]{datetime}
\setlength\parindent{0pt}
\newcommand{\Li}[2]{\operatorname{Li}_{#1}\left(#2\right)}
\newcommand{\Cl}[2]{\operatorname{Cl}_{#1}\left(#2\right)}
\newcommand{\genstirlingI}[3]{%
    \genfrac{[}{]}{0pt}{#1}{#2}{#3}%
}
\newcommand{\genstirlingII}[3]{%
    \genfrac{\{}{\}}{0pt}{#1}{#2}{#3}%
}
\newcommand{\stirlingI}[2]{\genstirlingI{}{#1}{#2}}
\newcommand{\dstirlingI}[2]{\genstirlingI{0}{#1}{#2}}
\newcommand{\tstirlingI}[2]{\genstirlingI{1}{#1}{#2}}
\newcommand{\stirlingII}[2]{\genstirlingII{}{#1}{#2}}
\newcommand{\dstirlingII}[2]{\genstirlingII{0}{#1}{#2}}
\newcommand{\tstirlingII}[2]{\genstirlingII{1}{#1}{#2}}


\begin{document}
	\maketitle 
	
	% \begin{abstract}
	% 	\noindent
	% 	Trong bài viết này, tác giả sẽ thực hiện tính toán bằng Toán học và viết mã thực thi tính toán tích phân cho tổng $\displaystyle \sum\limits_{i=0}^n \gamma_i f_i^{\alpha_i}(\beta_i x)$ với $\alpha_i \in \mathbb{N},\,\beta_i \in \mathbb{Z}_{\ne 0},\,\gamma_i \in \mathbb{Z}_{\ne 0}$ và $f_i$ là một trong các hàm lượng giác $\sin,\,\cos,\,\tan,\,\cot,\,\sec,\,\csc,\,\arcsin,\,$ $\arccos,\,\arctan,\,{\rm arccot},\,{\rm arcsec},\,{\rm arccsc}$. 

	% 	\noindent\textbf{Từ khóa:} integral, trigonometry.
	% \end{abstract}

	% --------------- Bài 1 ---------------
	\begin{tcolorbox}[breakable]
    	\begin{baitoan}[Mở rộng đề thi THPTQG Toán 2025]
			($\sum = 1$ điểm) Có $m\in \mathbb{N^\star}$ ngăn trong $1$ giá để sách được đánh số thứ tự $1,\,2,\,\ldots,\,m$ và $n\in \mathbb{N^\star}$ nquyển sách khác nhau. Xếp $n$ quyển sách này vào $m$ ngăn đó, các quyển sách được xếp thẳng đứng thành $1$ hàng ngang với gáy sách quay ra ngoài ở mỗi ngăn. Khi đã xếp xong $n$ quyển sách, hai cách xếp được gọi là giống nhau nếu chúng thỏa mãn đồng thời $2$ điều kiện sau: (i) với từng ngăn, số lượng quyển sách ở ngăn đó là như nhau trong cả hai cách xếp; (ii) với từng ngăn, thứ tự từ trái sang phải của các quyển sách được xếp là như nhau trong cả hai cách xếp. Đếm số cách xếp đôi một khác nhau nếu: \begin{enumerate}[label=(\alph*)]
				\item (0.5 điểm) mỗi ngăn có ít nhất $1$ quyển sách.
				\item (0.5 điểm) mỗi ngăn có thể không có quyển nào.
			\end{enumerate}
		\end{baitoan}
	\end{tcolorbox}

	\textbf{Lời giải. }
	\begin{enumerate}[label=(\alph*)]
		\item Trường hợp $n$ cuốn sách là giống nhau, khi đó bài toán trở thành bài toán chia kẹo Euler (phiên bản \ref{euler_division_ver1}) nên có $\displaystyle {n-1 \choose m-1}$ cách xếp. Tuy nhiên, các cuốn sách là khác nhau, nên với mỗi cách xếp nói trên, thì mỗi hoán vị lại tạo ra một cách xếp với; có tất cả $n!$ hoán vị nên số cách xếp sách thỏa mãn đề bài sẽ bằng $n!\displaystyle {n-1 \choose m-1}$.
		\item Trường hợp $n$ cuốn sách là giống nhau, khi đó bài toán trở thành bài toán chia kẹo Euler (phiên bản \ref{euler_division_ver2}) nên có $\displaystyle {n+m-1 \choose m-1}$ cách xếp. Tuy nhiên, các cuốn sách là khác nhau, nên với mỗi cách xếp nói trên, thì mỗi hoán vị lại tạo ra một cách xếp với; có tất cả $n!$ hoán vị nên số cách xếp sách thỏa mãn đề bài sẽ bằng $n!\displaystyle {n+m-1 \choose m-1}$.
	\end{enumerate}

	% --------------- Bài 2 ---------------
	\begin{tcolorbox}[breakable]
    	\begin{baitoan}[Đẳng thức Vandermonde]
			($\sum = 2$ điểm) 
			\begin{enumerate}[label=(\alph*)]
				\item (1 điểm) Chứng minh đẳng thức Vandermonde $$\sum\limits_{i=0}^r {m \choose i} {n \choose r-i} = {m+n \choose r},\quad\forall m,\,n,\,r\in \mathbb{N}$$ bằng hai cách: (i) (0.5 điểm) phương pháp tổ hợp; (ii) (0.5 điểm) phương pháp đại số thông qua việc tính hệ số của $x^r$ trong khai triển của $(1+x)^m(1+x)^n$.
				\item (1 điểm) Chứng minh đẳng thức Vandermonde tổng quát $${\sum\limits_{i=1}^p n_i \choose m} = \sum\limits_{\sum\limits_{i=1}^p k_i = m} \prod\limits_{i=1}^p {n_i \choose k_i},\,\text{i.e., }{n_1 + \cdots + n_p \choose m} = \sum\limits_{k_1+\cdots+k_p = m} {n_1\choose k_1}{n_2\choose k_2}\cdots{n_p\choose k_p}$$ bằng hai cách: (i) (0.5 điểm) phương pháp tổ hợp; (ii) (0.5 điểm) phương pháp đại số.
			\end{enumerate}
		\end{baitoan}
	\end{tcolorbox}

	\textbf{Lời giải. }
	\begin{enumerate}[label=(\alph*)]
		\item (i) Xét bài toán: lớp học có $m$ nam và $n$ nữ, cần chọn ra $r$ bạn; tính số cách. Ta đếm bằng hai cách: \begin{itemize}
			\item chọn ra $r$ bạn từ $m+n$ bạn, có tất cả $\displaystyle {m+n \choose r}$ cách;
			\item chọn ra $i$ nam từ $m$ nam thì có $\displaystyle {m \choose i}$ cách, sau đó chọn $r-i$ nữ từ $n$ nữ thì có $\displaystyle {n \choose r-i}$ cách, nên chọn $r$ bạn trong đó có $i$ nam sẽ có $\displaystyle{m \choose i}{n \choose r-i}$ cách, xét tất cả các trường hợp của $i$ thì có tất cả $\displaystyle\sum\limits_{i=0}^r {m \choose i} {n \choose r-i}$ cách.
		\end{itemize}
		Từ hai cách đếm nói trên, ta có $\displaystyle\sum\limits_{i=0}^r {m \choose i} {n \choose r-i} = {m+n \choose r}$.

		(ii) Áp dụng Nhị thức Newton ta có 
		$$\displaystyle(1+x)^m = \sum\limits_{i=0}^m{m \choose i}x^i,\quad (1+x)^n  = \sum\limits_{i=0}^n{n\choose i}x^i,\quad (1+x)^{m+n}  = \sum\limits_{i=0}^{m+n}{{m+n}\choose i}x^i.$$

		Trong khai triển của $(1+x)^m(1+x)^n$, hạng tử $x^r$ được tạo thành từ việc nhân hạng tử $x^i$ trong khai triển của $(1+x)^m$ với hạng tử $x^{r-i}$ trong khai triển của $(1+x)^n$. Do đó hệ số của $x^r$ trong khai triển $(1+x)^m(1+x)^n$ sẽ bằng $\displaystyle\sum\limits_{i=0}^r{m \choose i}{n\choose i}$, bằng với hệ số của $x^r$ trong khai triển $(1+x)^{m+n}$ là $\displaystyle{m+n\choose r}$.
		\item (i) Xét bài toán: lớp được chia thành $p$ đội thi, mỗi đội thi có $n_i$ thành viên; cần chọn ra $m$ thành viên để tham gia vòng thi, tính số cách. Ta đếm bằng hai cách:
		\begin{itemize}
			\item chọn ra $m$ bạn từ tất cả $\displaystyle\sum\limits_{i=1}^p n_i$ bạn, có $\displaystyle {n_1 + \cdots + n_p \choose m}$ cách;
			\item chọn ra $k_j$ bạn từ mỗi đội $j$ có $n_j$ bạn thì có $\displaystyle {n_j \choose k_j}$ cách, nhưng phải đảm bảo tổng cộng chọn được đúng $m$ bạn, như vậy có $
			\displaystyle\sum\limits_{k_1+\cdots+k_p = m} {n_1\choose k_1}{n_2\choose k_2}\cdots{n_p\choose k_p}$ cách.
		\end{itemize}
		Từ hai cách đếm nói trên, ta có $\displaystyle{n_1 + \cdots + n_p \choose m} = \sum\limits_{k_1+\cdots+k_p = m} {n_1\choose k_1}{n_2\choose k_2}\cdots{n_p\choose k_p}$.

		(ii) Áp dụng Nhị thức Newton ta có
		$$(1+x)^{n_1+n_2+\cdots+n_p} = \sum\limits_{i=0}^{n_1+n_2+\cdots+n_p} {n_1+n_2+\cdots+n_p \choose i}x^i,\quad (1+x)^{n_j} = \sum\limits_{i=0}^{n_j} {n_j \choose i}x^i.$$

		Trong khai triển của $\displaystyle \prod\limits_{j=1}^p (1+x)^{n_j}$, hạng tử $x^m$ được tạo thành từ việc nhân các hạng tử $x^j$ trong khai triển của $(1+x)^{n_j}$ lại với nhau sao cho số mũ là $m$. Do đó hệ số của $x^m$ trong khai triển $\displaystyle \prod\limits_{j=1}^p (1+x)^{n_j}$ sẽ bằng $\displaystyle\sum\limits_{k_1+\cdots+k_p = m} {n_1\choose k_1}{n_2\choose k_2}\cdots{n_p\choose k_p}$, bằng với hệ số của $x^m$ trong khai triển $(1+x)^{n_1+n_2+\cdots+n_p}$ là $\displaystyle {n_1 + \cdots + n_p \choose m}$.
	\end{enumerate}

	% --------------- Bài 3 ---------------
	\begin{tcolorbox}[breakable]
    	\begin{baitoan}[Đẳng thức Hockey-stick]
			($\sum = 2$ điểm) Chứng minh đẳng thức Hockey-stick 
			$$\sum\limits_{i=r}^n {i \choose r} = \sum\limits_{j=0}^{n-r} {j+r \choose r} = \sum\limits_{j=0}^{n-r} {j+r \choose j} = {n+1 \choose n-r},\quad\forall n,\,r\in \mathbb{N},\,n\ge r.$$
			bằng 4 cách: 
			\begin{enumerate}[label=(\alph*)]
				\item (0.5 điểm) quy nạp;
				\item (0.5 điểm) biến đổi đại số;
				\item (0.5 điểm) phương pháp tổ hợp;
				\item (0.5 điểm) sử dụng hàm sinh bằng cách tính hệ số của $x^r$ trong biểu thức $\displaystyle \sum\limits_{i=r}^n (x+1)^i = (x+1)^r+(x+1)^{r+1}+\cdots+(x+1)^n$.
			\end{enumerate}
		\end{baitoan}
	\end{tcolorbox}

	\textbf{Lời giải. }Đặt $j = i-r$ thì ta có $\displaystyle \sum\limits_{i=r}^n {i \choose r} = \sum\limits_{j=0}^{n-r} {j+r \choose r} = \sum\limits_{j=0}^{n-r} {j+r \choose j}$, ngoài ra thì $\displaystyle{n+1 \choose n-r} = {n+1 \choose r+1}$ nên ta chỉ cần chứng minh $\displaystyle \sum\limits_{i=r}^n {i \choose r} = {n+1 \choose r+1}$ là đủ.
	\begin{enumerate}[label=(\alph*)]
		\item Ta sẽ cố định $r\in \mathbb{N}$ và chứng minh quy nạp theo $n$. Trường hợp $n=0$ thì $\displaystyle \sum\limits_{i=0}^0 {i \choose 0} = 1 = {1 \choose 1}$ nên đẳng thức đúng với $n=0$. Trường hợp $n=1$ thì đẳng thức cần chứng minh trở thành $\displaystyle\sum\limits_{i=r}^1 {i \choose r} = {2\choose r+1}$ với mọi $0\le r\le 1$, nếu $r=0$ thì $\displaystyle\sum\limits_{i=0}^1 {i \choose 0} = 2 = {2\choose 1}$, còn nếu $r=1$ thì $\displaystyle\sum\limits_{i=1}^1 {i \choose 1} = 1 = {2\choose 2}$, nên đẳng thức cũng đúng với $n=1$. Giả sử đẳng thức đúng tới $n$, khi đó kết hợp với công thức Pascal thì $$\displaystyle \sum\limits_{i=r}^{n+1} {i \choose r} = \sum\limits_{i=r}^{n} {i \choose r} + {n+1 \choose r} = {n+1 \choose r+1} + {n+1 \choose r} = {n+2 \choose r+1}$$ nên đẳng thức cũng đúng với $n+1$. Theo nguyên lý quy nạp ta có điều phải chứng minh.
		\item Để ý rằng $\displaystyle {r \choose r} = 1  = {r+1 \choose r+1}$ nên $\displaystyle \sum\limits_{i=r}^n {i \choose r} = {r+1 \choose r+1} + \sum\limits_{i=r+1}^n {i \choose r}$. Áp dụng liên tiếp công thức Pascal, ta được $\displaystyle {r+1 \choose r+1} + \sum\limits_{i=r+1}^n {i \choose r} = {r+2 \choose r+1} + \sum\limits_{i=r+2}^n {i \choose r} = {r+3 \choose r+1} + \sum\limits_{i=r+3}^n {i \choose r} = \cdots = {n \choose r+1} + {n \choose r} = {n+1 \choose r+1}$.
		\item Xét bài toán: lớp học có $n+1$ bạn, cần chọn ra $r+1$ bạn đi thi Olympic. Ta đếm bằng hai cách: \begin{itemize}
			\item chọn $r+1$ bạn từ $n+1$ bạn thì có $\displaystyle {n+1\choose r+1}$ cách;
			\item ta sắp xếp $n+1$ trong lớp theo số thứ tự từ $1$ đến $n+1$, bây giờ chọn đội tuyển $r+1$ người thì ta quan tâm đến bạn có số thứ tự lớn nhất được chọn vào đội, khi đó bạn có số thứ tự lớn nhất trong mỗi cách chọn sẽ luôn lớn hơn hoặc bằng $r+1$; giả sử bạn có số thứ tự lớn nhất được chọn là bạn thứ $i+1\,(r \le i \le n)$ thì $r$ bạn còn lại phải được chọn từ $i$ bạn có số thứ tự nhỏ hơn (tức từ bạn thứ $1$ đến bạn thứ $i$), nên có $\displaystyle {i \choose r}$ cách; do đó số cách tất cả sẽ là $\displaystyle \sum\limits_{i=r}^n {i \choose r}$.
		\end{itemize}Từ hai cách đếm nói trên ta có điều phải chứng minh.
		\item Vì $\displaystyle {i \choose r}$ là hệ số của $x^r$ trong khai triển của $(1+x)^i$ nên vế trái chính là hệ số của $x^r$ trong khai triển của hàm sinh $\displaystyle P(x) = \sum\limits_{i=r}^n (1+x)^i$, ta ký hiệu $\displaystyle \sum\limits_{i=r}^n {i \choose r} = \left[x^r\right]P(x)$. Mặt khác, áp dụng công thức tổng cấp số nhân thì $P(x) = (1+x)^r \dfrac{(1+x)^{n-r+1}-1}{(1+x)-1} = \dfrac{(1+x)^{n+1} - (1+x)^r}{x}$ nên suy ra $\displaystyle \sum\limits_{i=r}^n {i \choose r} = \left[x^r\right]P(x) = \left[x^{r+1}\right]\left((1+x)^{n+1} - (1+x)^r\right) = {n+1\choose r+1}$ (do $(1+x)^r$ không chứa hạng tử mũ $r+1$).
	\end{enumerate}

	% --------------- Bài 4 ---------------
	\begin{tcolorbox}[breakable]
    	\begin{baitoan}[Công thức Pascal]
			($\sum = 2.5$ điểm) 
			\begin{enumerate}[label=(\alph*)]
				\item (1 điểm) Chứng minh công thức Pascal $\displaystyle {n-1\choose k} + {n-1 \choose k-1} = {n\choose k}$ bằng 2 cách: (i) (0.5 điểm) biến đổi đại số; (ii) (0.5 điểm) phương pháp tổ hợp.
				\item (0.5 điểm) Tìm công thức cho hệ số của $\displaystyle \prod\limits_{i=1}^m x_i^{k_i} = x_1^{k_1}x_2^{k_2}\cdots x_m^{k_m}$ trong khai triển của $\left(\sum\limits_{i=1}^m x_i\right)^n = (x_1 + x_2 + \cdots + x_m)^n$.
				\item (1 điểm) Đặt hệ số ở câu (b) là $c(m,\,n,\,k_1,\,k_2,\,\ldots,\,k_m)$, viết lại đẳng thức ở câu (a) với $m=2$ theo ký hiệu này. Chứng minh công thức Pascal tổng quát 
				\begin{align*}
					&c(m,\,n - 1,\,k_1 - 1,\,k_2,\ldots,\,k_m) + c(m,\,n - 1,\,k_1,\,k_2 - 1,\,\ldots,\,k_m) + \cdots + c(m,\,n - 1,\,k_1,\,k_2,\,\ldots,\,k_m - 1)\\
					&= c(m,\,n,\,k_1,\,k_2,\,\ldots,\,k_m),\ \forall m\in\mathbb{N},\,m\ge2,\ k_i\in\mathbb{N}^\star,\ \forall i\in[m],\ \sum_{i=1}^m k_i = n.
				\end{align*}
			\end{enumerate}
		\end{baitoan}
	\end{tcolorbox}

	\textbf{Lời giải. }
	\begin{enumerate}[label=(\alph*)]
		\item (i)  Ta có $${n-1\choose k} + {n-1 \choose k-1} = \dfrac{(n-1)!}{k!(n-k-1)!} + \dfrac{(n-1)!}{(k-1)!(n-k)!} = \dfrac{(n-1)!}{k! (n-k)!} \left(n-k + k\right) = \dfrac{n!}{k! (n-k)!} = {n\choose k}.$$
		
		(ii) Xét bài toán: trong lớp có $n$ học sinh, cần chọn ra $k$ bạn; tính số cách. Ta đếm bằng hai cách:
		\begin{itemize}
			\item chọn $k$ bạn từ $n$ bạn, có $\displaystyle {n\choose k}$ cách;
			\item nếu trong $k$ bạn được chọn, có bạn lớp trưởng thì cần chọn $k-1$ bạn còn lại trong $n-1$ bạn, có ${n-1 \choose k-1}$ cách; nếu trong $k$ bạn được chọn, không có bạn lớp trưởng thì cần chọn $k$ bạn trong $n-1$ bạn (bỏ qua bạn lớp trưởng), có $\displaystyle{n-1\choose k}$ cách; tổng có $\displaystyle{n-1 \choose k-1} +{n-1\choose k}$ cách.
		\end{itemize}
		Từ hai cách đếm trên, ta có điều phải chứng minh.
		\item Theo định lý đa thức (\href{https://en.wikipedia.org/wiki/Multinomial_theorem}{multinomial theorem}), hệ số của $\displaystyle \prod\limits_{i=1}^m x_i^{k_i}$ trong khai triển của $\displaystyle\left(\sum\limits_{i=1}^m x_i\right)^n$ bằng $${n\choose k_1,\,k_2,\,\ldots,\,k_m} = \dfrac{n!}{k_1! k_2! \cdots k_m!}.$$
		\item Hệ số nhị thức $\displaystyle {n\choose k}$ tương ứng với hệ số của $x_1^kx_2^{n-k}$ trong khai triển $(x_1+x_2)^n$, tức $\displaystyle {n\choose k} = c(2,\,n,\,k,\,n-k)$ nên công thức ở câu (a) khi viết lại sẽ là $$c(2,\,n-1,\,k,\,n-k-1) + c(2,\,n-1,\,k-1,\,n-k) = c(2,\,n,\,k,\,n-k).$$ Trong công thức Pascal tổng quát, ta thực hiện đếm bằng hai cách: \begin{itemize}
			\item vế phải $c(m,\,n,\,k_1,\,k_2,\,\ldots,\,k_m)$ là số cách xếp $n$ vật vào $m$ hộp khác nhau sao cho hộp $i$ chứa đúng $k_i$ vật;
			\item vế trái, xét một vật đặc biệt trong $n$ vật; ta xét $m$ trường hợp, đối với trường hợp $i$ thì vật đặc biệt được đặt vào hộp $i$, ta cần xếp $n-1$ vật còn lại sao cho hộp $i$ có $k_i-1$ vật và các hộp $j$ còn lại có $k_j$ vật; khi đó số cách là $c(m,\,n-1,\,k_1,\,k_2,\,\ldots,\,k_i-1,\,k_{i+1},\,\ldots,\,k_m)$.
		\end{itemize}
		Từ hai cách đếm, ta có điều phải chứng minh.
	\end{enumerate}

	% --------------- Bài 5 ---------------
	\begin{tcolorbox}[breakable]
    	\begin{baitoan}[Bài toán chia kẹo Euler] 
			($\sum = 3$ điểm)
			\begin{enumerate}[label=(\alph*)]
				\item (0.5 điểm) Phát biểu hai phiên bản của bài toán chia kẹo Euler.
				\item (1.5 điểm) Đếm số nghiệm nguyên của phương trình $\displaystyle \sum\limits_{i=1}^m x_i = n$ với điều kiện: (i) (0.25 điểm) $x_i \geq 0,\,\forall i\in [m]$; (ii) (0.25 điểm) $x_i \geq 1,\,\forall i\in [m]$; (iii) (0.5 điểm) $2 \mid x_i,\,\forall i\in [m]$; (iv) (0.5 điểm) $2 \nmid x_i,\,\forall i\in [m]$.
				\item (1 điểm) Gọi $a(m,\,n),\,b(m,\,n),\,c(m,\,n),\,d(m,\,n)$ lần lượt là số nghiệm nguyên tương ứng ở ý trước, thiết lập các công thức đệ quy, quy hoạch động cho chúng.
			\end{enumerate}
		\end{baitoan}
	\end{tcolorbox}

	\textbf{Lời giải. }

	\begin{enumerate}[label=(\alph*)]
		\item Phát biểu hai phiên bản của bài toán chia kẹo Euler. 
		\begin{enumerate}[label=\arabic*.]
			\item Cho hai số nguyên dương $n$ và $k$, số bộ $k$ số nguyên dương có tổng bằng $n$ bằng với số tập con có $k-1$ phần tử của tập hợp gồm $n-1$ phần tử, bằng với \begin{align}
				{n-1 \choose k-1}. \label{euler_division_ver1}\tag{\thebaitoan.\alph{enumi}.\arabic{enumii}}
				\end{align}
			\item Cho hai số nguyên dương $n$ và $k$, số bộ $k$ số nguyên không âm có tổng bằng $n$ bằng với số multiset có $k-1$ phần tử của tập hợp gồm $n+1$ phần tử, tương đương với số multiset có $n$ phần tử của tập hợp gồm $k$ phần tử, bằng với \begin{align}
				{\displaystyle \left(\!\!{\binom {n+1}{k-1}}\!\!\right)} = {\displaystyle \left(\!\!{\binom {k}{n}}\!\!\right)} = {n+k-1 \choose k-1}. \label{euler_division_ver2}\tag{\thebaitoan.\alph{enumi}.\arabic{enumii}}
			\end{align}
		\end{enumerate}
		\item (i) Đây là phiên bản \ref{euler_division_ver2} nên số nghiệm sẽ bằng $\displaystyle {n+m-1 \choose m-1}$.
		
		(ii) Đây là phiên bản \ref{euler_division_ver1} nên số nghiệm sẽ bằng $\displaystyle {n-1 \choose m-1}$.

		(iii) Đặt $x_i = 2y_i$ với mọi $i\in [m]$. Khi đó bài toán quy về đếm số nghiệm nguyên không âm của phương trình $y_1 + y_2 +\ldots + y_m = \dfrac{n}{2}$. Như vậy, nếu $n$ lẻ thì phương trình vô nghiệm, còn nếu $n$ chẵn thì theo phiên bản \ref{euler_division_ver2}, số nghiệm sẽ bằng $\displaystyle {n/2 +m-1 \choose m-1}$.

		(iv) Đặt $x_i = 2y_i+1$ với mọi $i\in [m]$. Khi đó bài toán quy về đếm số nghiệm nguyên không âm của phương trình $y_1 + y_2 +\ldots + y_m = \dfrac{n-m}{2}$. Như vậy, nếu $n-m$ lẻ thì phương trình vô nghiệm, còn nếu $n-m$ chẵn thì theo phiên bản \ref{euler_division_ver2}, số nghiệm sẽ bằng $\displaystyle {(n-m)/2 +m-1 \choose m-1}$.
		\item (i) Ta xét 2 trường hợp: \begin{itemize}
			\item nếu $x_m = 0$ thì phương trình trở thành $x_1 + x_2 + \ldots + x_{m-1} = n$, số nghiệm là $a(m-1,\,n)$;
			\item nếu $x_m > 0$, đặt $x_m' = x_m - 1 \ge 0$ thì phương trình trở thành $x_1 + x_2 + \ldots + x_{m-1} + x_m' = n-1$, số nghiệm là $a(m,\,n-1)$.
		\end{itemize}
		Như vậy ta có $a(m,\,n) = a(m-1,\,n) + a(m,\,n-1)$. Trường hợp cơ sở $a(m,\,0) = 1$ (có một nghiệm duy nhất là $x_1 = x_2 = \cdots = x_m = 0$); $a(1,\,n) = 1$ (có một nghiệm duy nhất là $x_1 = n$).
		(ii) Ta xét 2 trường hợp: \begin{itemize}
			\item nếu $x_m = 1$ thì phương trình trở thành $x_1 + x_2 + \ldots + x_{m-1} = n-1$, số nghiệm là $b(m-1,\,n-1)$;
			\item nếu $x_m > 1$, đặt $x_m' = x_m - 1 \ge 1$ thì phương trình trở thành $x_1 + x_2 + \ldots + x_{m-1} + x_m' = n-1$, số nghiệm là $b(m,\,n-1)$.
		\end{itemize}
		Như vậy ta có $b(m,\,n) = b(m-1,\,n-1) + b(m,\,n-1)$. Trường hợp cơ sở $b(m,\,n) = 0$ nếu $n < m$; $b(m,\,m) = 1$ (có một nghiệm duy nhất là $x_1 = x_2 = \cdots = x_m = 1$).
		(iii) Ta xét 2 trường hợp: \begin{itemize}
			\item nếu $x_m = 0$ thì phương trình trở thành $x_1 + x_2 + \ldots + x_{m-1} = n$, số nghiệm là $c(m-1,\,n)$;
			\item nếu $x_m \geq 2$, đặt $x_m' = x_m - 2 \ge 0$ và là số chẵn thì phương trình trở thành $x_1 + x_2 + \ldots + x_{m-1} + x_m' = n-2$, số nghiệm là $c(m,\,n-2)$.
		\end{itemize}
		Như vậy ta có $c(m,\,n) = c(m-1,\,n) + c(m,\,n-2)$. Trường hợp cơ sở $c(m,\,n) = 0$ nếu $n$ lẻ; $c(m,\,0) = 1$; $c(1,\,n) = 1$ nếu $n$ chẵn và $c(1,\,n) = 0$ nếu $n$ lẻ.
		(iii) Ta xét 2 trường hợp: \begin{itemize}
			\item nếu $x_m = 1$ thì phương trình trở thành $x_1 + x_2 + \ldots + x_{m-1} = n-1$, số nghiệm là $d(m-1,\,n-1)$;
			\item nếu $x_m \geq 3$, đặt $x_m' = x_m - 2 \ge 1$ và là số lẻ thì phương trình trở thành $x_1 + x_2 + \ldots + x_{m-1} + x_m' = n-2$, số nghiệm là $d(m,\,n-2)$.
		\end{itemize}
		Như vậy ta có $d(m,\,n) = d(m-1,\,n-1) + d(m,\,n-2)$. Trường hợp cơ sở $d(m,\,n) = 0$ nếu $n$ và $m$ khác tính chẵn lẻ hoặc $n < m$; $d(m,\,m) = 1$ (có một nghiệm duy nhất là $x_1 = x_2 = \cdots = x_m = 1$).
	\end{enumerate}

	
	% --------------- Bài 6 ---------------
	\begin{tcolorbox}[breakable]
    	\begin{baitoan}[Đếm số các đơn thức monic] 
			($\sum = 2$ điểm)
			Một đơn thức monic của $n \in \mathbb{Z^+}$ biến $x_1,\,x_2,\,\ldots,\,x_n$ là một biểu thức toán học có dạng $\displaystyle \prod_{i=1}^n x_i^{a_i} = x_1^{a_1}x_2^{a_2}\cdots x_n^{a_n}$ với $a_i \in \mathbb{N},\,\forall i\in [n]$; bậc $d = \displaystyle\sum\limits_{i=1}^n a_i = a_1 + a_2 + \cdots + a_n$.
			\begin{enumerate}[label=(\alph*)]
				\item (0.5 điểm) Đếm số đơn thức monic bậc $d$ của $n$ biến $x_1,\,x_2,\,\ldots,\,x_n$.
				\item (0.5 điểm) Đếm số đơn thức monic bậc không vượt quá $d$ của $n$ biến $x_1,\,x_2,\,\ldots,\,x_n$.
				\item (1 điểm) Gọi $a(n,\,d),\,b(n,\,d)$ lần lượt là số đơn thức thỏa mãn ở hai câu trước, thiết lập các công thức đệ quy, quy hoạch động cho chúng.
			\end{enumerate}
		\end{baitoan}
	\end{tcolorbox}

	\textbf{Lời giải. }
	\begin{enumerate}[label=(\alph*)]
		\item Số đơn thức monic bậc $d$ cần tìm bằng với số nghiệm nguyên không âm của phương trình $a_1 + a_2 + \cdots + a_n = d$, bằng $\displaystyle {d+n-1 \choose n-1}$.
		\item Số đơn thức monic bậc $i\,(0 \leq i \leq d)$ bằng với số nghiệm nguyên không âm của phương trình $a_1 + a_2 + \cdots + a_n = i$, bằng $\displaystyle {i+n-1 \choose n-1}$. Suy ra số đơn thức monic bậc không vượt quá $d$ bằng với $\displaystyle\sum\limits_{i=0}^d {i+n-1 \choose n-1} = {d+n \choose n}$, theo đẳng thức Hockey-stick.
		\item (i) Ta xét 2 trường hợp: \begin{itemize}
			\item nếu $a_n = 0$ thì phương trình trở thành $a_1 + a_2 + \ldots + a_{n-1} = d$, số nghiệm là $a(n-1,\,d)$;
			\item nếu $a_m > 0$, đặt $a_m' = a_m - 1 \ge 0$ thì phương trình trở thành $a_1 + a_2 + \ldots + a_{m-1} + a_m' = d-1$, số nghiệm là $a(n,\,d-1)$.
		\end{itemize}
		Như vậy ta có $a(n,\,d) = a(n-1,\,d) + a(n,\,d-1)$. Trường hợp cơ sở $a(n,\,0) = 1$ (chỉ có một đơn thức là 1, mọi $a_i = 0$); $a(1,\,d) = 1$ (chỉ có một đơn thức là $x_1^{d_1}$).

		(ii) Ta có $\displaystyle b(n,\,d) = \sum\limits_{i=0}^d a(n,\,i)$ nên $b(n,\,d) = b(n,\,d-1) + a(n,\,d)$. Trường hợp cơ sở $b(n,\,0)=1$ (chỉ có đơn thức bậc 0); $b(1,\,d) = d+1$ (các đơn thức là $1,\,x_1,\,x_1^2,\,\ldots,\,x_1^d$).
	\end{enumerate}



	% --------------- Bài 7 ---------------
	% --------------- Bài 8 ---------------
	% --------------- Bài 9 ---------------
	% --------------- Bài 10 ---------------







	\begin{tcolorbox}[breakable]
    	\begin{baitoan}[Mở rộng \cite{shahriari2021invitation}, P1.3.4, p. 22]
			($\sum = 7$ điểm)
			Cho $n\in\mathbb{N}^\star$. Xét 1 dải bìa cứng $1\times n$. Chúng ta có 1 số lượng lớn $m$ loại mảnh có kích thước $\{1\times i\}_{i=2}^{m+1} = 1\times2,1\times3,\ldots,1\times(m + 1)$. Cho $f(n)$ là số cách ta có thể lát cho dải bìa cứng bằng các mảnh của mình.
			\begin{enumerate}[label=(\alph*)]
				\item (0.5 điểm) Tính $f(1),\,f(2),\,f(3),\,\ldots,\,f(10)$. 
				\item (0.5 điểm) Chứng minh công thức truy hồi cho $f(n)$: $f(n) = \sum_{i=2}^{m + 1} f(n - i),\,\forall n\in\mathbb{N}^\star$.
				\item (1 điểm) Làm lại 2 ý trước nếu $m$ loại mảnh đổi thành $1\times1,\,1\times2,\,\ldots,\,1\times m$ cho $g(n)$ là số cách lát thỏa mãn. 
				\item (5 điểm) Gọi $f(m,\,n),\,g(m,\,n)$ lần lượt là số cách ta có thể lát cho dải bìa cứng $m\times n$ bằng $m$ loại mảnh có kích thước $\{1\times i\}_{i=2}^{m+1},\,\{1\times i\}_{i=1}^m$. Thực hiện các yêu cầu: (i) (0.5 điểm) tính $f(2,\,2),\,f(3,\,2),\,f(4,\,2),\,f(5,\,2)$; (ii) (1 điểm) thiết lập công thức truy hồi cho $f(n,\,2) = f(2,\,n),\,g(n,\,2) = g(2,\,n)$; (iii) (3.5 điểm) thiết lập công thức truy hồi cho $f(m,\,n),\,g(m,\,n)$.
			\end{enumerate}
		\end{baitoan}
	\end{tcolorbox}

	\textbf{Lời giải. }

	\begin{tcolorbox}[breakable]
    	\begin{baitoan}[Số Stirling]
			($\sum = 1$ điểm)
			\begin{enumerate}[label=(\alph*)]
				\item (0.5 điểm) \cite[P6.2.7]{shahriari2021invitation} Tìm \& chứng minh công thức dạng
				\begin{equation*}
					\stirlingI{n}{n - 2} = \binom{n}{*} + *\binom{n}{*},\ \forall n\in\mathbb{N},\,n\ge2.
				\end{equation*}
				\item (0.5 điểm) \cite[P6.1.13]{shahriari2021invitation} Tìm \& chứng minh công thức dạng
				\begin{equation*}
					\stirlingII{n}{n - 2} = \binom{n}{*} + *\binom{n}{*},\ \forall n\in\mathbb{N},\,n\ge2.
				\end{equation*} 
			\end{enumerate}
		\end{baitoan}
	\end{tcolorbox}

	\textbf{Lời giải. }

	\begin{tcolorbox}[breakable]
    	\begin{baitoan}[Định lý Euler và thuật toán Havel$-$Hakimi]
			($\sum = 1.5$ điểm)
			\begin{enumerate}[label=(\alph*)]
				\item (0.5 điểm) Phát biểu định lý Euler và định lý về thuật toán Havel$-$Hakimi. Hai định lý này áp dụng cho các loại đồ thị nào?
				\item (0.5 điểm) \cite[P10.1.13, p. 368]{shahriari2021invitation} Chứng minh 1 dãy $d_1,\,d_2,\,\ldots,\,d_n$ là 1 graphic sequence khi \& chỉ khi dãy $n - d_n - 1,\,\ldots,\,n - d_2 - 1,\,n - d_1 - 1$ là graphic; áp dụng kiểm tra $9,\,9,\,9,\,9,\,9,\,9,\,9,\,9,\,8,\,8,\,8$ có phải là graphic squence không. 	
				\item (0.5 điểm) Sử dụng thuật toán Havel$-$Hakimi để kiểm tra dãy $9,\,9,\,9,\,9,\,9,\,9,\,9,\,9,\,8,\,8,\,8$ có phải là graphic sequence hay không.
			\end{enumerate}
		\end{baitoan}
	\end{tcolorbox}

	\textbf{Lời giải. }

	\begin{enumerate}[label=(\alph*)]
		\item \textbf{Định lý Euler. }Xét $G = (V,\,E)$ là một đồ thị tổng quát. Gọi $\{d_i\}_{i=1}^{|V|}$ là dãy bậc của các đỉnh. Khi đó 
		\begin{equation*}
			\sum_{i=1}^{|V|} d_i = 2|E|.
		\end{equation*}

		\textbf{Thuật toán Havel-Hakimi. }Xét 2 dãy số nguyên không âm. Giả sử dãy đầu tiên được sắp xếp theo thứ tự không tăng và $t_s \geq 1$: (a) $s,t_1,\ldots,t_s,d_1,\ldots,d_n$. (b) $t_1 - 1,t_2 - 1,\ldots,t_s - 1,d_1,\ldots,d_n$. Khi đó dãy (a) là dãy graphic khi và chỉ khi dãy (b) là dãy graphic (sau khi sắp xếp lại).

		\item Xét đồ thị đơn với dãy bậc $d_1,\,d_2,\,\ldots,\,d_p$, ta thêm vào các cạnh để trở thành đồ thị đầy đủ và bỏ đi những cạnh ban đầu. Khi đó đồ thị thu được là đồ thị với dãy bậc $p - d_p - 1,\,\ldots,\,p-d_2-1,\,p-d_1-1$ nên ta có điều phải chứng minh.
		
		Áp dụng, dãy $9,\,9,\,9,\,9,\,9,\,9,\,9,\,9,\,8,\,8,\,8$ là một dãy graphic khi và chỉ khi dãy $2,\,2,\,2,\,1,\,1,\,1,\,1,\,1,\,1,\,1,\,1$ là một dãy graphic. Theo thuật toán Havel-Hakimi, điều đó tương đương với $1,\,1,\,1,\,1,\,1,\,1,\,1,\,1,\,1,\,1$ cũng là một dãy graphic; điều này đúng vì tồn tại đồ thị có dãy bậc là dãy này, chẳng hạn xét đồ thị 10 đỉnh trong đó các đỉnh được chia thành 5 cặp, mỗi đỉnh trong một cặp nối với nhau (hay nói cách khác là 5 đoạn thẳng rời nhau).

		\item \textbf{Bước 1:} Xóa phần tử đầu (9) và trừ 1 vào 9 phần tử tiếp theo. Dãy mới thu được là $(8, 8, 8, 8, 8, 8, 8, 7, 7, 8)$. Sắp xếp lại ta có:
			$$S_1 = (8, 8, 8, 8, 8, 8, 8, 8, 7, 7).$$

			\textbf{Bước 2:} Xóa phần tử đầu (8), trừ 1 vào 8 phần tử tiếp theo. Dãy mới thu được là $(7, 7, 7, 7, 7, 7, 7, 6, 7)$. Sắp xếp lại ta có:
			$$S_2 = (7, 7, 7, 7, 7, 7, 7, 7, 6).$$

			\textbf{Bước 3:} Xóa phần tử đầu (7), trừ 1 vào 7 phần tử tiếp theo. Dãy mới thu được là $(6, 6, 6, 6, 6, 6, 6, 6)$. Dãy đã được sắp xếp:
			$$S_3 = (6, 6, 6, 6, 6, 6, 6, 6).$$

			\textbf{Bước 4:} Xóa phần tử đầu (6), trừ 1 vào 6 phần tử tiếp theo. Dãy mới thu được là $(5, 5, 5, 5, 5, 5, 6)$. Sắp xếp lại ta có:
			$$S_4 = (6, 5, 5, 5, 5, 5, 5).$$

			\textbf{Bước 5:} Xóa phần tử đầu (6), trừ 1 vào 6 phần tử tiếp theo. Dãy mới thu được là $(4, 4, 4, 4, 4, 4)$. Dãy đã được sắp xếp:
			$$S_5 = (4, 4, 4, 4, 4, 4).$$

			\textbf{Bước 6:} Xóa phần tử đầu (4), trừ 1 vào 4 phần tử tiếp theo. Dãy mới thu được là $(3, 3, 3, 3, 4)$. Sắp xếp lại ta có:
			$$S_6 = (4, 3, 3, 3, 3).$$

			\textbf{Bước 7:} Xóa phần tử đầu (4), trừ 1 vào 4 phần tử tiếp theo. Dãy mới thu được là $(2, 2, 2, 2)$. Dãy đã được sắp xếp:
			$$S_7 = (2, 2, 2, 2).$$

			\textbf{Bước 8:} Xóa phần tử đầu (2), trừ 1 vào 2 phần tử tiếp theo. Dãy mới thu được là $(1, 1, 2)$. Sắp xếp lại ta có:
			$$S_8 = (2, 1, 1).$$

			\textbf{Bước 9:} Xóa phần tử đầu (2), trừ 1 vào 2 phần tử tiếp theo. Dãy mới thu được là $(0, 0)$:
			$$S_9 = (0, 0).$$

			Một dãy gồm toàn các số 0 là một dãy graphic (tương ứng với một đồ thị không có cạnh nào) nên dãy đã cho cũng là một dãy graphic.
	\end{enumerate}

	\begin{tcolorbox}[breakable]
    	\begin{baitoan}[Vài đồ thị đơn đặc biệt \cite{valiente2002algorithms}]
			($\sum = 3$ điểm)
			\begin{enumerate}[label=(\alph*)]
				\item (0.5 điểm) Chứng minh số cạnh tối đa của 1 đồ thị đơn có $n\in\mathbb{N}^\star$ đỉnh bằng $\dfrac{1}{2}n(n - 1)$. Khi đẳng thức xảy ra, đồ thị đơn hữu hạn đó được gọi là đồ thị gì? Viết định nghĩa, vẽ, tìm số cạnh \& dãy bậc của đồ thị đó.
				\item (0.5 điểm) Viết định nghĩa, vẽ, tìm số cạnh \& dãy bậc của đồ thị đường đi -- path graph $P_n$.
				\item (0.5 điểm) Viết định nghĩa, vẽ, tìm số cạnh \& dãy bậc của đồ thị chu trình -- cycle graph $C_n$.
				\item (0.5 điểm) Viết định nghĩa, vẽ, tìm số cạnh \& dãy bậc của đồ thị bánh xe -- wheel graph $W_{n+1}$.
				\item (0.5 điểm) Viết định nghĩa, vẽ, tìm số cạnh \& dãy bậc khả dĩ của đồ thị chính quy -- regular graph.
				\item (0.5 điểm) Tìm số cạnh \& dãy bậc của đồ thị 2 phía đầy đủ -- complete bipartite graph $K_{m,\,n}$.
			\end{enumerate}
		\end{baitoan}
	\end{tcolorbox}

	\textbf{Lời giải. }

	\begin{enumerate}[label=(\alph*)]
		\item Đồ thị đơn có tối đa cạnh khi là đồ thị đầy đủ, mỗi đỉnh được nối với $n-1$ đỉnh còn lại và do mỗi cạnh nối giữa hai đỉnh sẽ được tính 2 lần nên số cạnh sẽ bằng $\dfrac{n(n-1)}{2}$. Một đồ thị vô hướng $G = (V,\,E)$ được gọi là đồ thị đầy đủ $K_n$ nếu $(v,\,w)\in E$, $\forall v,w\in V$ với $v\ne w$. Số cạnh bằng $\dfrac{n(n-1)}{2}$. Dãy bậc gồm $n$ số $n-1$.
		\item Đồ thị đường đi $P_n$ là một đồ thị đơn vô hướng liên thông $G = (V,\,E)$ với $|V| = |E| + 1$ và ta có thể vẽ tất cả các đỉnh trên cùng một đường thẳng. Số cạnh bằng $n-1$. Dãy bậc gồm hai số $1$ và $n-2$ số 2.
		\item Đồ thị chu trình $C_n$ là một đồ thị vô hướng liên thông $G = (V,\,E)$ với $|V| = |E|$ và nếu nó có thể được vẽ sao cho mọi đỉnh đều nằm trên 1 đường tròn. Số cạnh bằng $n$. Dãy bậc gồm $n$ số $2$.
		\item Đồ thị bánh xe $W_{n+1}$ là một đồ thị vô hướng liên thông $G = (V,\,E)$ và nếu nó có thể được vẽ sao cho tất cả trừ 1 đỉnh nằm trên 1 đường tròn có tâm tại đỉnh duy nhất, được kết nối với tất cả các đỉnh khác. Số cạnh bằng $2n$. Dãy bậc gồm một số $n$ và $n$ số 3: $n,\,3,\,3,\,\ldots,\,3$.
		\item Đồ thị chính quy là một đồ thị vô hướng $G = (V,\,E)$ với $\deg v = \deg w$, $\forall v,w\in V$. Số cạnh bằng $\dfrac{nk}{2}$ với $k$ là bậc của các đỉnh. Dãy bậc gồm $n$ số $k$.
		\item Đồ thị hai phía đầy đủ $K_{m,\,n}$ vừa là đồ thị hai phía (có các đỉnh có thể được phân chia thành 2 tập $X,\,Y$ theo cách mà tất cả các cạnh đều có 1 đầu trong $X$ \& 1 đầu khác trong $Y$), trong đó mỗi đỉnh trong mỗi phần được nối với toàn bộ đỉnh ở phần bên kia. Số cạnh bằng $m \times n$. Dãy bậc gồm $m$ số $n$ và $n$ số $m$.
	\end{enumerate}

	\bibliography{refs}

	% \vspace*{\fill}
	% \hfill \text{Cập nhật đến: \date{\today}.}
\end{document}