\subsection{Nhị thức Newton}

\begin{tcolorbox}[breakable]
    \begin{baitoan}~
        \label{pb:w01:02}

        \begin{enumerate}
            \item[(a)] {Khai triển $(a + b + c)^n$.}
            \item[(b)] {Khai triển $(a + b + c + d)^n$.} 
            \item[(c)] {Khai triển $\displaystyle\left(\sum\limits_{i=1}^m a_i\right)^n$.} 
        \end{enumerate}
    \end{baitoan}
\end{tcolorbox}

\textbf{Lời giải. }

\begin{enumerate}
    \item[(a)] {$\displaystyle \left(a+(b+c)\right)^n = \sum\limits_{i = 0}^n {n \choose i} \cdot a^{n-i} \cdot (b+c)^i = \sum\limits_{i = 0}^n {n \choose i} \cdot a^{n-i} \cdot \left(\sum\limits_{j = 0}^{i} {i \choose j} \cdot b^{i-j} \cdot c^j\right)$ \\ $\displaystyle = \sum\limits_{i = 0}^n \sum\limits_{j = 0}^{i} {n \choose i} {i \choose j} a^{n-i} b^{i-j} c^j$.}
    \item[(b)] {$\displaystyle (a + b + c + d)^n = \left(a+(b+c+d)\right)^n = \sum\limits_{i = 0}^n {n \choose i} \cdot a^{n-i} \cdot (b+c+d)^i$ \\ $\displaystyle = \sum\limits_{i = 0}^n {n \choose i} \cdot a^{n-i} \cdot \left(\sum\limits_{j = 0}^i \sum\limits_{k = 0}^{j} {i \choose j} {j \choose k} b^{i-j} c^{j-k} d^k\right) = \sum\limits_{i = 0}^n \sum\limits_{j = 0}^i \sum\limits_{k = 0}^j {n \choose i} {i \choose j} {j \choose k} a^{n-i} b^{i-j} c^{j-k} d^k$.} 
    \item[(c)] {Ta sẽ chứng minh quy nạp theo $m$ rằng $$\displaystyle\left(\sum\limits_{i=1}^m a_i\right)^n = \sum\limits_{i_1 = 0}^n \sum\limits_{i_2 = 0}^{i_1} \cdots \sum\limits_{i_{m-1} = 0}^{i_{m-2}} \left({n \choose i_1} {i_1 \choose i_2} \cdots {i_{m-2} \choose i_{m-1}} a_1^{n-i_1} a_2^{i_1-i_2} \cdots a_{m-1}^{i_{m-2}-i_{m-1}}a_m^{i_{m-1}}\right).$$
    
    Với $m = 1,\,2,\,3$ thì đẳng thức trên đúng.
    
    Giả sử đẳng thức trên đúng tới $m = M$, tức ta đã có 
    
    $$\displaystyle\left(\sum\limits_{i=1}^M a_i\right)^n = \sum\limits_{i_1 = 0}^n \sum\limits_{i_2 = 0}^{i_1} \cdots \sum\limits_{i_{M-1} = 0}^{i_{M-2}} \left({n \choose i_1} {i_1 \choose i_2} \cdots {i_{M-2} \choose i_{M-1}} a_1^{n-i_1} a_2^{i_1-i_2} \cdots a_{M-1}^{i_{M-2}-i_{M-1}}a_M^{i_{M-1}}\right).$$

    Ta cần chứng minh đẳng thức trên cũng đúng với $m = M+1$, tức cần chứng minh 
    $$\displaystyle\left(\sum\limits_{i=1}^{M+1} a_i\right)^n = \sum\limits_{i_1 = 0}^n \sum\limits_{i_2 = 0}^{i_1} \cdots \sum\limits_{i_{M} = 0}^{i_{M-1}} \left({n \choose i_1} {i_1 \choose i_2} \cdots {i_{M-1} \choose i_{M}} a_1^{n-i_1} a_2^{i_1-i_2} \cdots a_{M}^{i_{M-1}-i_{M}}a_{M+1}^{i_{M}}\right).$$

    Thật vậy, áp dụng Nhị thức Newton và giả thiết quy nạp ở trên ta được, 
    \begin{align*}
        \displaystyle\left(\sum\limits_{i=1}^{M+1} a_i\right)^n 
        &= \sum\limits_{i_1 = 0}^n {n \choose i_1}a_1^{n-i_1}\left(\sum\limits_{i=2}^{M+1}a_i\right)^{i_1} \\
        &= \sum\limits_{i_1 = 0}^n {n \choose i_1}a_1^{n-i_1} \left(\sum\limits_{i_2 = 0}^{i_1} \sum\limits_{i_3 = 0}^{i_2} \cdots \sum\limits_{i_{M} = 0}^{i_{M-1}} \left({i_1 \choose i_2} {i_2 \choose i_3} \cdots {i_{M-1} \choose i_{M}} a_2^{i_1-i_2} a_3^{i_2-i_3} \cdots a_{M+1}^{i_{M}}\right)\right) \\
        &= \sum\limits_{i_1 = 0}^n \sum\limits_{i_2 = 0}^{i_1} \cdots \sum\limits_{i_{M} = 0}^{i_{M-1}} \left({n \choose i_1} {i_1 \choose i_2} \cdots {i_{M-1} \choose i_{M}} a_1^{n-i_1} a_2^{i_1-i_2} \cdots a_{M}^{i_{M-1}-i_{M}}a_{M+1}^{i_{M}}\right).
    \end{align*}

    Như vậy đẳng thức cũng đúng với $m = M+1$. Theo nguyên lý quy nạp toán học ta có điều phải chứng minh. 
    } 
    % \item[(d)] {$\displaystyle z^n = (a+b\text{i})^n = \sum\limits_{k = 0}^n {n \choose k} a^{n-k}(b\text{i})^k$} 
\end{enumerate}