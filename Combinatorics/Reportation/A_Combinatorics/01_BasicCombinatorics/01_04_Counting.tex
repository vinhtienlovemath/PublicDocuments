\subsection{Quy tắc đếm}

% \begin{tcolorbox}[breakable]
%     \begin{baitoan}
%         Cho $n\in\mathbb{Z^+}$. Chứng minh rằng số cách khác nhau để đặt đúng đắn $n$ dấu ngoặc mở và $n$ dấu ngoặc đóng bằng số Catalan thứ $n$ $$C_n = \dfrac{1}{n+1}{2n \choose n} = \dfrac{(2n)!}{n!(n+1)!}.$$
%     \end{baitoan}
% \end{tcolorbox}

% \textbf{Lời giải. }


\begin{tcolorbox}[breakable]
    \begin{baitoan}\label{pb:w02:02}
        Cho $m,\,n \in \mathbb{Z^+},\,2m \leq n$. Đếm số các dãy $a_1,\,a_2,\,\ldots,\,a_n$ chứa $m$ số 0 và $n-m$ số 1, thỏa mãn không có hai phần tử nào kề nhau đều là số 0.
    \end{baitoan}
\end{tcolorbox}

\textbf{Lời giải. }Xếp $m$ số 0 thành hàng ngang, khi đó sẽ có $m-1$ khoảng trống ở giữa (cần chèn vào tối thiểu một số 1) và 2 khoảng trống ở hai đầu. Như vậy số dãy thỏa mãn chính là số nghiệm nguyên của phương trình $a_1 + a_2 + \cdots + a_{m+1} = n-m$, với $a_1,\,a_{m+1} \geq 0$ và $a_i \geq 1\,(2 \leq i \leq m)$. Đặt $b_1 = a_1 + 1,\,b_{m+1}=a_{m+1}+1$ thì ta cần tìm số nghiệm nguyên dương của phương trình $b_1 + a_2 + \cdots + a_m + b_{m+1} = n - m + 2$. Theo bài toán chia kẹo Euler, phương trình có tất cả $\displaystyle {n-m+1 \choose m}$ cách, cũng là số các dãy $a_1,\,a_2,\,\ldots,\,a_n$ cần tìm.

\begin{tcolorbox}[breakable]
    \begin{baitoan}\label{pb:w02:03}
        Đặt $f(n)$ là số tập con của $[n]$. Chứng minh rằng $f(n) = 2^n$ với mọi $n$ nguyên dương.
    \end{baitoan}
\end{tcolorbox}

\textbf{Lời giải. }Trường hợp $n=1$, có hai tập con của $[1]$ là $\varnothing,\,\{1\}$ nên $f(1) = 2^1$.

Giả sử mệnh đề đúng tới $n = N$, tức ta đã có $f(N) = 2^N$. Xét các tập con của $[N+1]$, có hai loại
\begin{enumerate}
    \item[$\bullet$] các tập con của $[N]$ (không chứa phần tử $N+1$), có $2^N$ tập;
    \item[$\bullet$] các tập hợp là hợp của $\{N+1\}$ và từng tập con của $[N]$, cũng có $2^N$ tập.
\end{enumerate}

Như vậy $f(N+1) = 2^N + 2^N = 2^{N+1}$ nên mệnh đề cũng đúng với $n = N+1$. Theo nguyên lý quy nạp toán học, ta có điều phải chứng minh.

\begin{tcolorbox}[breakable]
    \begin{baitoan} \label{pb:w05:04}
        Giả sử rằng $n \in \mathbb{Z^+}$ người đưa $n$ chiếc mũ của họ cho mỗi người kiểm tra mũ. Đặt $f(n)$ là số cách trả lại các chiếc mũ, sao cho mỗi người có đúng 1 mũ và không ai nhận lại mũ của họ lúc ban đầu.

        \begin{enumerate}
            \item[(a)] Chứng minh rằng $\displaystyle f(n) = n! \sum\limits_{i=0}^n \dfrac{(-1)^i}{i!}$ với mọi số nguyên dương $n$.
            \item[(b)] Chứng minh rằng $f(n)$ là số nguyên gần nhất với $\dfrac{n!}{\mathrm{e}}$.
        \end{enumerate}
    \end{baitoan}
\end{tcolorbox}

\textbf{Lời giải. }

\begin{enumerate}
    \item[(a)] {
        Khi $n = 1$ thì $f(1) = 0$ (không có cách nào vì một người không thể nhận mũ của chính họ). Khi $n = 2$ thì $f(2) = 1$ (có duy nhất một cách trả lại các chiếc mũ thỏa mãn là đổi mũ cho nhau). Khi $n = 3$ thì $f(3) = 2$ (xét 3 người là $A,\,B,\,C$; giả sử $A$ nhận lại mũ của $B$, $B$ có thể nhận lại mũ của $A$ hoặc $C$, nhưng $B$ chỉ có thể nhận lại mũ của $C$ để $C$ không nhận lại mũ của chính mình, nên trường hợp này có 1 cách; tương tự nếu giả sử $A$ nhận lại mũ của $C$ thì cũng sẽ có 1 cách; như vậy tổng tất cả có 2 cách).

        Ta có nhận xét sau (\href{https://en.wikipedia.org/wiki/Derangement}{derangement}, \parencite{nqbh-combinatorics}): mỗi người có thể nhận được bất kỳ chiếc mũ nào trong số $n - 1$ chiếc mũ không phải của mình. Gọi chiếc mũ mà người $P_1$ nhận được là $h_i$ và xét đến chủ sở hữu của $h_i$: $P_i$ nhận được mũ của $P_1$, $h_1$ hoặc 1 chiếc mũ khác. Theo đó, bài toán chia thành 2 trường hợp có thể xảy ra:
        \begin{itemize}
            \item $P_i$ nhận được 1 chiếc mũ khác với $h_1$. Trường hợp này tương đương với việc giải bài toán với $n - 1$ người và $n - 1$ chiếc mũ vì đối với mỗi $n - 1$ người ngoài $P_1$ thì có đúng $1$ chiếc mũ trong số $n - 1$ chiếc mũ còn lại mà họ không được nhận (đối với bất kỳ $P_j$ nào ngoài $P_i$, chiếc mũ không được nhận là $h_j$, trong khi đối với $P_i$ thì là $h_1$). Một cách khác để thấy điều này là đổi tên $h_1$ thành $h_i$, trong đó sự sắp xếp rõ ràng hơn: đối với bất kỳ $j$ nào từ $2$ đến $n$, $P_j$ không thể nhận được $h_j$.
            \item $P_i$ nhận được $h_1$. Trong trường hợp này, bài toán được rút gọn thành $n - 2$ người và $n - 2$ mũ, vì $P_1$ nhận được mũ của $h_i$ và $P_i$ nhận được mũ của $h_1$, về cơ bản là loại cả hai ra khỏi việc xem xét thêm.
        \end{itemize}

        Đối với mỗi $n - 1$ mũ mà $P_1$ có thể nhận được, số cách mà $P_2,\,\ldots,\,P_n$ có thể nhận được mũ là tổng số đếm của 2 trường hợp. Từ đây ta có
        \begin{equation*}
            f(1) = 0,\,f(2) = 1,\,f(n) = (n - 1)(f(n - 1) + f(n - 2)),\,\forall n\in\mathbb{Z^+},\,n\geq 3.
        \end{equation*}         

        Từ công thức truy hồi trên, ta có $f(n) - nf(n-1) = -f(n-1) + (n-1)f(n-2)$ nên truy hồi ngược về thì $f(n) - nf(n-1) = (-1)^{n-2}\left(f(2) - f(1)\right) = (-1)^n,\,\forall n\in \mathbb{Z^+},\,n\geq 2$. Suy ra $\dfrac{f(n)}{n!} = \dfrac{f(n-1)}{(n-1)!} + \dfrac{(-1)^n}{n!}$, truy hồi ngược về thì được $\displaystyle\dfrac{f(n)}{n!} = \dfrac{f(1)}{1!} + \sum\limits_{i=2}^n \dfrac{(-1)^i}{i!} = \sum\limits_{i=0}^n \dfrac{(-1)^i}{i!}$. Như vậy $\displaystyle f(n) = n! \sum\limits_{i=0}^n \dfrac{(-1)^i}{i!}$ với mọi số nguyên dương $n$.
    }
    \item[(b)] Theo công thức khai triển chuỗi Maclaurin của hàm số $\mathrm{e}^x$ tại $x=-1$, ta có $$\left|\dfrac{n!}{\mathrm{e}} - f(n)\right| = \left|n! \sum\limits_{i=0}^\infty \dfrac{(-1)^i}{i!} - n! \sum\limits_{i=0}^n \dfrac{(-1)^i}{i!}\right| = \left|n!\sum\limits_{i=n+1}^\infty \dfrac{(-1)^i}{i!}\right|.$$
    
    \textbf{Trường hợp 1. }$n$ chẵn. Khi đó $$n!\sum\limits_{i=n+1}^\infty \dfrac{(-1)^i}{i!} = -\dfrac{1}{n+1} +\dfrac{1}{(n+1)(n+2)} -\dfrac{1}{(n+1)(n+2)(n+3)} +\cdots$$ Vì $-\dfrac{1}{(n+1)\cdots(n+2k-1)}+\dfrac{1}{(n+1)\cdots(n+2k-1)(n+2k)} = \dfrac{-n-2k+1}{(n+1)\cdots(n+2k-1)(n+2k)} < 0$ nên $\displaystyle n!\sum\limits_{i=n+1}^\infty \dfrac{(-1)^i}{i!} < 0$. Mặt khác, vì $\dfrac{1}{(n+1)\cdots(n+2k)}-\dfrac{1}{(n+1)\cdots(n+2k-1)(n+2k+1)} = \dfrac{n+2k}{(n+1)\cdots(n+2k-1)(n+2k+1)} > 0$ nên $\displaystyle n!\sum\limits_{i=n+1}^\infty \dfrac{(-1)^i}{i!} > -\dfrac{1}{n+1}$. Từ đó suy ra $-\dfrac{1}{n+1} < \displaystyle n!\sum\limits_{i=n+1}^\infty \dfrac{(-1)^i}{i!} < 0$.

    \textbf{Trường hợp 2. }$n$ lẻ. Bằng việc ghép cặp tương tự trường hợp trên, ta chứng minh được $0 < \displaystyle n!\sum\limits_{i=n+1}^\infty \dfrac{(-1)^i}{i!} < \dfrac{1}{n+1}$.

    Như vậy trong mọi trường hợp thì $\displaystyle \left|n!\sum\limits_{i=n+1}^\infty \dfrac{(-1)^i}{i!}\right| < \dfrac{1}{n+1}$. Do đó $\left|\dfrac{n!}{\mathrm{e}} - f(n)\right| < \dfrac{1}{n+1} \leq \dfrac{1}{2}$ với mọi số nguyên dương $n$ nên $f(n)$ là số nguyên gần nhất với $\dfrac{n!}{\mathrm{e}}$.
\end{enumerate}

\begin{tcolorbox}[breakable]
    \begin{baitoan}\label{pb:w05:03}
        Cho số nguyên dương $n$. Đặt $f(n)$ là số các tập con của $[n]$ không chứa bất kỳ hai số nguyên dương liên tiếp nào.

        \begin{enumerate}
            \item[(a)] Tính $f(1),\,f(2),\,f(3),\,f(4)$.
            \item[(b)] Chứng minh rằng $f(n) = f(n-1)+f(n-2)$ với mọi $n$ nguyên dương, $n \geq 3$.
            \item[(c)] Chứng minh rằng $f(n) = \dfrac{1}{\sqrt{5}}\left(\tau^{n+2} - \overline{\tau}^{n+2}\right)$, với $\tau = \dfrac{1+\sqrt{5}}{2},\,\overline{\tau} = \dfrac{1-\sqrt{5}}{2}$.
        \end{enumerate}
    \end{baitoan}
\end{tcolorbox}

\textbf{Lời giải. }Gọi $S_n$ là tập hợp các tập con của $[n]$ không chứa bất kỳ hai số nguyên dương liên tiếp nào.

\begin{enumerate}
    \item[(a)] {
        $S_1 = \left\{\varnothing,\,\{1\}\right\}$ nên $f(1) = 2$;

        $S_2 = \left\{\varnothing,\,\{1\},\,\{2\}\right\}$ nên $f(2) = 3$;

        $S_3 = \left\{\varnothing,\,\{1\},\,\{2\},\,\{3\},\,\{1,\,3\}\right\}$ nên $f(3) = 5$;

        $S_4 = \left\{\varnothing,\,\{1\},\,\{2\},\,\{3\},\,\{4\},\,\{1,\,3\},\,\{1,\,4\},\,\{2,\,4\}\right\}$ nên $f(4) = 8$.
    }
    \item[(b)] {
        Ta phân $S_n$ thành hai tập:
        \begin{enumerate}
            \item[$\bullet$] tập những tập con của $[n]$ không chứa phần tử $n$: các tập như vậy chính là các tập con của $[n-1]$, nên sẽ có tất cả $f(n-1)$ tập.
            \item[$\bullet$] tập những tập con của $[n]$ có chứa phần tử $n$: các tập như vậy sẽ không chứa phần tử $n-1$, nên sẽ là các tập con của $[n-2]$ hợp với tập $\{n\}$, nên sẽ có tất cả $f(n-2)$ tập. 
        \end{enumerate}

        Như vậy $f(n) = f(n-1) + f(n-2)$.
    }
    \item[(c)] {
        Xét phương trình đặc trưng $\tau^2 = \tau + 1$, phương trình có nghiệm $\tau = \dfrac{1+\sqrt{5}}{2}$ và $\overline{\tau} = \dfrac{1-\sqrt{5}}{2}$.

        Suy ra $f(n) = A\cdot \tau^n + B\cdot \overline{\tau}^n,\,\forall n\in \mathbb{Z^+}$. Vì $A\cdot \tau + B\cdot \overline{\tau} = f(1) = 2$ và $A\cdot \tau^2 + B\cdot \overline{\tau}^2 = f(2) = 3$, nên $A = \dfrac{5+3\sqrt{5}}{10}$ và $B = \dfrac{5-3\sqrt{5}}{10}$. Suy ra $f(n) = \dfrac{5+3\sqrt{5}}{10} \cdot \tau^n + \dfrac{5-3\sqrt{5}}{10} \cdot \overline{\tau}^n = \dfrac{1}{\sqrt{5}}\left(\tau^{n+2} - \overline{\tau}^{n+2}\right)$.
    }
\end{enumerate}