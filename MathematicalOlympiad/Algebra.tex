\newpage
\section{Đề thi Đại số}
\begin{tcolorbox}[title=\textbf{Bài toán B.1.}]
    Cho $a$ là một số thực, $A$ là ma trận phụ thuộc vào $a$
    $$A = \begin{pmatrix}[]{cccc}
        1 & a+1 & a+2 & 0 \\
        a+3 & 1 & 0 & a+2 \\
        a+2 & 0 & 1 & a+1 \\
        0 & a+2 & a+3 & 1 
    \end{pmatrix}.$$

    \begin{enumerate}
        \item[(a)] Tìm hạng của ma trận $A$ khi $a = -1$.
        \item[(b)] Tìm tất cả các số thực $a$ sao cho $A$ có định thức dương.
        \item[(c)] Biện luận số chiều của không gian nghiệm của hệ phương trình tuyến tính $AX = 0$ theo $a$ (ở đây $X$ là vector cột ứng với các tọa độ lần lượt là $x,\,y,\,z,\,t$). 
    \end{enumerate}
\end{tcolorbox}

\textbf{Lời giải.}

\begin{enumerate}
    \item[(a)] {
        Khi $a = -1$ thì 
        \begin{align*}
            A 
            & = \begin{pmatrix}[]{cccc}
                1 & 0 & 1 & 0 \\
                2 & 1 & 0 & 1 \\
                1 & 0 & 1 & 0 \\
                0 & 1 & 2 & 1 
            \end{pmatrix} \sim \begin{pmatrix}[]{cccc}
                1 & 0 & 1 & 0 \\
                2 & 1 & 0 & 1 \\
                0 & 1 & 2 & 1 \\
                1 & 0 & 1 & 0 
            \end{pmatrix} \sim \begin{pmatrix}[]{cccc}
                1 & 0 & 1 & 0 \\
                0 & 1 & -2 & 1 \\
                0 & 1 & 2 & 1 \\
                0 & 0 & 0 & 0 
            \end{pmatrix} \sim \begin{pmatrix}[]{cccc}
                1 & 0 & 1 & 0 \\
                0 & 1 & -2 & 1 \\
                0 & 0 & 4 & 0 \\
                0 & 0 & 0 & 0 
            \end{pmatrix}.
        \end{align*}

        Từ đây suy ra $\text{r}(A) = 3$.
    }
    \item[(b)] {
        Theo khai triển Laplace, ta có 

        \begin{align*}
            \det (A) 
            & = \begin{vmatrix}[]{ccc}
                1 & 0 & a+2 \\
                0 & 1 & a+1 \\
                a+2 & a+3 & 1 
            \end{vmatrix} - (a+1)\begin{vmatrix}[]{ccc}
                a+3 & 0 & a+2 \\
                a+2 & 1 & a+1 \\
                0 & a+3 & 1 
            \end{vmatrix} + (a+2)\begin{vmatrix}[]{ccc}
                a+3 & 1 & a+2 \\
                a+2 & 0 & a+1 \\
                0 & a+2 & 1 
            \end{vmatrix} \allowdisplaybreaks \\ 
            & = \left(\begin{vmatrix}[]{cc}
                1 & a+1 \\
                a+3 & 1 
            \end{vmatrix} + (a+2)\begin{vmatrix}[]{cc}
                0 & 1 \\
                a+2 & a+3 
            \end{vmatrix}\right) \\
            & \quad - (a+1)\left(\begin{vmatrix}[]{cc}
                a+3 & 0 \\
                a+2 & 1 
            \end{vmatrix} - (a+3)\begin{vmatrix}[]{cc}
                a+3 & a+2 \\
                a+2 & a+1 
            \end{vmatrix}\right) \allowdisplaybreaks \\
            & \quad + (a+2)\left(\begin{vmatrix}[]{cc}
                a+3 & 1 \\
                a+2 & 0 
            \end{vmatrix} - (a+2)\begin{vmatrix}[]{cc}
                a+3 & a+2 \\
                a+2 & a+1 
            \end{vmatrix}\right) \\   
            & = \Big(1-(a+3)(a+1) - (a+2)^2\Big) - (a+1)(a+3) - (a+2)^2 \\ 
            & \quad + \Big((a+1)(a+3) - (a+2)^2\Big)\begin{vmatrix}[]{cc}
                a+3 & a+2 \\
                a+2 & a+1 
            \end{vmatrix} \\
            & = -4a^2 - 16a - 13 - \begin{vmatrix}[]{cc}
                a+3 & a+2 \\
                a+2 & a+1 
            \end{vmatrix} \\
            & = -4a^2 - 16a - 12 \\
            & = -4(a+1)(a+3).
        \end{align*}

        Như vậy $\det (A) > 0 \Leftrightarrow -4(a+1)(a+3) > 0 \Leftrightarrow -3 < a < -1$.
    }
    \item[(c)] {
        Với $a = -1$, theo câu (a) ta có $\text{r}(A) = 3$ nên số chiều của không gian nghiệm bằng $4 - \text{r}(A) = 1$.

        Với $a = -3$, ta có $$A = \begin{pmatrix}[]{cccc}
            1 & -2 & -1 & 0 \\
            0 & 1 & 0 & -1 \\
            -1 & 0 & 1 & -2 \\
            0 & -1 & 0 & 1 
        \end{pmatrix} \sim \begin{pmatrix}[]{cccc}
            1 & -2 & -1 & 0 \\
            0 & 1 & 0 & -1 \\
            0 & -2 & 0 & -2 \\
            0 & 0 & 0 & 0 
        \end{pmatrix} \sim \begin{pmatrix}[]{cccc}
            1 & -2 & -1 & 0 \\
            0 & 1 & 0 & -1 \\
            0 & 0 & 0 & -4 \\
            0 & 0 & 0 & 0 
        \end{pmatrix},$$ từ đây suy ra $\text{r}(A) = 3$ nên số chiều của không gian nghiệm bằng $4 - \text{r}(A) = 1$.

        Với $a \ne -1,\,-3$ thì $\det (A) \ne 0$ nên hệ phương trình $AX = 0$ có nghiệm duy nhất nên số chiều của không gian nghiệm bằng 0.
    }
\end{enumerate}

\begin{tcolorbox}[title=\textbf{Bài toán B.2.}]
    Cho $A,\,B$ là hai ma trận vuông $$A = \begin{pmatrix}[]{cc}
        0 & 2 \\
        1 & 0 
    \end{pmatrix},\,B = \begin{pmatrix}[]{cc}
        0 & -2 \\
        -1 & 0 
    \end{pmatrix}.$$

    \begin{enumerate}
        \item[(a)] Tìm một ma trận thực $P$ có cấp bằng 2, sao cho $P^{-1}AP$ là ma trận đường chéo.
        \item[(b)] Tìm một ma trận thực $R$ có cấp bằng 2, định thức bằng 1, sao cho $R^{-1}AR = B$.
    \end{enumerate}
\end{tcolorbox}

\textbf{Lời giải.}

\begin{enumerate}
    \item[(a)] {
        Xét ma trận $P = \begin{pmatrix}[]{cc}
            \sqrt{2} & -\sqrt{2} \\
            1 & 1
        \end{pmatrix}$ có $\det (P) = 2\sqrt{2} \ne 0$ nên $P$ khả nghịch.
        
        Khi đó $$P^{-1} = \dfrac{1}{\det (P)} \cdot \text{adj}(P) = \dfrac{1}{2\sqrt{2}}\begin{pmatrix}[]{cc}
            1 & \sqrt{2} \\
            -1 & \sqrt{2} 
        \end{pmatrix},$$ trong đó $\text{adj}(P)$ là ma trận phụ hợp của ma trận $P$. 

        Kiểm tra bằng phép nhân ma trận, ta thấy $$P^{-1}AP = \begin{pmatrix}[]{cc}
            \sqrt{2} & 0 \\
            0 & -\sqrt{2} 
        \end{pmatrix}$$ là ma trận đường chéo.
    }
    \item[(b)] {
        Xét ma trận $Q = \begin{pmatrix}[]{cc}
            1 & -2 \\ 
            1 & -1
        \end{pmatrix}$ có $\det (Q) = 1 \ne 0$ nên $Q$ khả nghịch.
        
        Khi đó $$Q^{-1} = \dfrac{1}{\det (Q)} \cdot \text{adj}(Q) = \begin{pmatrix}[]{cc}
            -1 & 2 \\
            -1 & 1
        \end{pmatrix},$$ trong đó $\text{adj}(Q)$ là ma trận phụ hợp của ma trận $Q$. 

        Kiểm tra bằng phép nhân ma trận, ta thấy $$Q^{-1}AQ = \begin{pmatrix}[]{cc}
            0 & -2 \\
            -1 & 0 
        \end{pmatrix} = B.$$
    } 
\end{enumerate}

\textbf{Nhận xét. }Câu (a) là một bài toán chéo hóa ma trận cơ bản. Câu (b) giải được bằng cách đồng nhất hệ số và giải hệ phương trình. Ma trận $Q$ cần tìm có dạng tổng quát $\begin{pmatrix}[]{cc}
    a & b \\
    c & d 
\end{pmatrix}$, khi đó điều kiện đầu tiên là $\det(Q) = 1$ hay $ad - bc = 1$. Ngoài ra, $Q^{-1} = \dfrac{1}{\det (Q)} \cdot \text{adj}(Q) = \text{adj}(Q) = \begin{pmatrix}[]{cc}
    d & -b \\
    -c & a 
\end{pmatrix}$. Khi đó bằng phép nhân ma trận, ta có $Q^{-1}AQ = \begin{pmatrix}[]{cc}
    2cd-ab & 2d^2-b^2 \\
    a^2-2c^2 & ab-2cd 
\end{pmatrix}$ nên từ đó ta phải có $$\begin{pmatrix}[]{cc}
    2cd-ab & 2d^2-b^2 \\
    a^2-2c^2 & ab-2cd 
\end{pmatrix} = \begin{pmatrix}[]{cc}
    0 & -2 \\
    -1 & 0 
\end{pmatrix}.$$ 

Ta chọn $a,\,b,\,c,\,d$ thỏa mãn hệ phương trình $$\begin{cases}
    ad - bc = 1 \\
    2cd - ab = 0 \\
    2d^2 - b^2 = -2 \\
    a^2 - 2c^2 = -1 \\
    ad - 2cd = 0
\end{cases},$$
từ đó tìm được một bộ $(a,\,b,\,c,\,d)$ thỏa mãn hệ phương trình trên là $(1,\,-2,\,1,\,-1)$.


