\subsection{Phân hoạch nguyên}
\begin{tcolorbox}[breakable]
    \begin{baitoan}[\cite{shahriari2021invitation}, P.7.1.7, p. 238]\label{pb:w08:22}
        Cho $n,j\in\mathbb{N}^\star$, $j\le n$. Cho ${\cal A}$ là tập hợp các phân hoạch của $n$ có ít nhất $j$ phần bằng $1$. $|{\cal A}|$ là gì theo hàm phân hoạch $p$?
    \end{baitoan}
\end{tcolorbox}

\textbf{Lời giải. }Hàm sinh cho hàm phân hoạch $p(n)$ là $$P(x) = \sum\limits_{n=0}^\infty p(n) x^n = \prod\limits_{i=1}^\infty \dfrac{1}{1-x^i}.$$ Hàm sinh cho các phân hoạch có ít nhất $j$ phần bằng $1$ là $$G(x) = \left((x^1)^j + (x^1)^{j+1}+\cdots\right)\prod\limits_{i=2}^\infty \dfrac{1}{1-x^i} = \dfrac{x^j}{1-x}\prod\limits_{i=2}^\infty \dfrac{1}{1-x^i} = x^jP(x) = \sum\limits_{m=0}^\infty p(m)x^{m+j}.$$

Số phân hoạch của $n$ cần tìm chính là hệ số của $x^n$ trong $G(x)$, nên $m+j = n$ hay $m = n-j$. Như vậy $|{\cal A}| = p(n-j)$.

\begin{tcolorbox}[breakable]
    \begin{baitoan}[\cite{shahriari2021invitation}, P.7.1.8, p. 238]\label{pb:w08:23}
        Cho $n,k\in\mathbb{N}^\star$ với $k\le n$. Cho ${\cal A}$ là tập hợp các phân hoạch của $n$ có ít nhất $1$ phần bằng $k$. Tìm $|{\cal A}|$ theo hàm phân hoạch $p$.
    \end{baitoan}
\end{tcolorbox}

\textbf{Lời giải. }Hàm sinh cho hàm phân hoạch $p(n)$ là $$P(x) = \sum\limits_{n=0}^\infty p(n) x^n = \prod\limits_{i=1}^\infty \dfrac{1}{1-x^i}.$$ Hàm sinh cho các phân hoạch có ít nhất $1$ phần bằng $k$ là $$G(x) = \left((x^k)^1 + (x^k)^{2}+\cdots\right)\prod\limits_{i=1,\,i\ne k}^\infty \dfrac{1}{1-x^i} = \dfrac{x^k}{1-x^k}\prod\limits_{i=1,\,i\ne k}^\infty \dfrac{1}{1-x^i} = x^kP(x).$$

Tương tự Bài \ref{pb:w08:22} ta có $|{\cal A}| = p(n-k)$.

\begin{tcolorbox}[breakable]
    \begin{baitoan}[\cite{shahriari2021invitation}, P.7.1.9, p. 238]\label{pb:w08:24}
        Xét $k\in\mathbb{N}^\star$ cố định. Chứng minh rằng $p_{n - k}(n) = p(k)$, $\forall n\in\mathbb{N}$, $n\ge2k$.
    \end{baitoan}
\end{tcolorbox}

\textbf{Lời giải. }Xét một phân hoạch của $n$ thành $n-k$ phần $\lambda = (\lambda_1,\,\ldots,\,\lambda_{n-k})$, trong đó $\displaystyle\sum\lambda_i = n$ và $\lambda_i \ge 1$. Xây dựng phân hoạch mới $\mu$ bằng cách đặt $\mu_i = \lambda_i - 1$. Khi đó tổng của phân hoạch mới là $\displaystyle\sum\limits_{i=1}^{n-k}\mu_i = k$ và $\mu_i \ge 0$. Do đó $\mu$ là một phân hoạch của $k$ thành $n-k$ phần không âm, từ đây có điều phải chứng minh. 

% \begin{tcolorbox}[breakable]
%     \begin{baitoan}[\cite{shahriari2021invitation}, P.7.1.11, p. 238]\label{pb:w08:25}
%         Xét $n,k\in\mathbb{N}^\star$, $k\le n$. Chứng minh rằng $p_k(n) = \sum_{i=0}^k p_i(n - k)$.
%     \end{baitoan}
% \end{tcolorbox}

% \textbf{Lời giải. }

% \begin{tcolorbox}[breakable]
%     \begin{baitoan}[\cite{shahriari2021invitation}, P.7.1.15, p. 238]\label{pb:w08:26}
%         Xét ${\cal A}$ là tập hợp tất cả các phân hoạch của $n$ mà không có phần nào bằng $1$. Chứng minh rằng $|{\cal A}| = p(n) - p(n - 1)$.
%     \end{baitoan}
% \end{tcolorbox}

% \textbf{Lời giải. }

% \begin{tcolorbox}[breakable]
%     \begin{baitoan}[\cite{shahriari2021invitation}, P.7.1.16, p. 239]\label{pb:w08:27}
%         Xét $n,k\in\mathbb{N}^\star$, $k > \frac{n}{2}$. Chứng minh rằng $p_k(n) = p_{k-1}(n - 1)$.
%     \end{baitoan}
% \end{tcolorbox}

% \textbf{Lời giải. }

% \begin{tcolorbox}[breakable]
%     \begin{baitoan}[\cite{shahriari2021invitation}, P.7.1.22, p. 239]\label{pb:w08:28}
%         Cho $n\in\mathbb{N}^\star$. (a) Chứng minh rằng $\lfloor\frac{n}{2}\rfloor + \lfloor\frac{n + 3}{2}\rfloor = n + 1$. (b) Cho $A$ là tập hợp các phân hoạch của $n + 6$ thành $3$ phần trong đó phần nhỏ nhất là $2$ hoặc lớn hơn. Đưa ra 1 chứng minh tổ hợp rằng $|A| = p_3(n) + p_2(n) + p_1(n)$. (c) Chứng minh rằng $p_3(n + 6) = p_3(n) + p_2(n) + p_1(n)+ p_2(n + 3)$. Công thức này có hợp lệ với $n = 0$ không? (d) Chứng minh rằng $p_3(n + 6) = p_3(n) + n + 3$. Công thức này có đúng với $n = 0$ không?
%     \end{baitoan}
% \end{tcolorbox}

% \textbf{Lời giải. }

% \begin{tcolorbox}[breakable]
%     \begin{baitoan}[\cite{shahriari2021invitation}, P7.2.1, p. 241]\label{pb:w08:29}
%         Cho $n,k\in\mathbb{N}^\star$, $k\le n$. Số lượng các thành phần của $n$ thành $k$ phần là bao nhiêu? Tổng số các thành phần của $n$ (thành bất kỳ số phần nào) là bao nhiêu?
%     \end{baitoan}
% \end{tcolorbox}

% \textbf{Lời giải. }

% \begin{tcolorbox}[breakable]
%     \begin{baitoan}[\cite{shahriari2021invitation}, P7.2.2, p. 241]\label{pb:w08:30}
%         Cho $n,k\in\mathbb{N}^\star$, $k\le n$. Sắp xếp $n$ que thành 1 hàng. (a) Có bao nhiêu cách bạn có thể đặt $k - 1$ ngôi sao vào giữa $n$ que sao cho không có $2$ ngôi sao nào cạnh nhau (\& không có ngôi sao nào trước hoặc sau các que)? (b) Có bao nhiêu cách bạn có thể đặt bất kỳ số lượng ngôi sao nào (bao gồm $0$ ngôi sao) vào giữa $n$ que sao cho không có $2$ ngôi sao nào cạnh nhau (\& không có ngôi sao nào trước hoặc sau các que)? (c) Có bất kỳ mối quan hệ nào giữa những câu hỏi này \& những câu hỏi của Prob. P7.2.1 không?
%     \end{baitoan}
% \end{tcolorbox}

% \textbf{Lời giải. }

% \begin{tcolorbox}[breakable]
%     \begin{baitoan}[\cite{shahriari2021invitation}, P7.2.3, p. 241]\label{pb:w08:31}
%        Cho $n\in\mathbb{N}^\star$. Số lượng các thành phần của $n$ (thành bất kỳ số phần nào) là bao nhiêu nếu mỗi phần đều lớn hơn $1$?
%     \end{baitoan}
% \end{tcolorbox}

% \textbf{Lời giải. }

% \begin{tcolorbox}[breakable]
%     \begin{baitoan}[\cite{shahriari2021invitation}, P7.2.5, p. 242]\label{pb:w08:32}
%         Cho $n,k\in\mathbb{N}^\star$, $k\le n$. Cho $s$ là số nghiệm nguyên dương của $\sum_{i=1}^k = n$. Chứng minh rằng $k!p_k(n)\ge s$. Kết luận rằng
%         \begin{equation*}
%             p_k(n)\ge\multiset{k}{n - k}.
%         \end{equation*}
%     \end{baitoan}
% \end{tcolorbox}

% \textbf{Lời giải. }
